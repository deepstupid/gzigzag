\documentclass[12pt]{article}
\usepackage[boldsans]{ccfonts}
\usepackage{euler}

% This document is not meant to be read by itself.
% It is a part of dp.wml (dp.html), included as images
% It exists because writing math in HTML is awful and looks even
% worse.

\begin{document}

\def\atom#1{{\mathcal #1}}

\thispagestyle{empty}
\pagestyle{empty}

% Page 1: basic atom interaction

We shall denote sets of quarks by calligraphic font: $\atom A$, $\atom B$.
Let $\atom X$ be the complete change from version A to version B.
A set $\atom A$ is atomical w.r.t. $\atom X$ if
$\atom A \subseteq \atom X$ and for all connections modified in $\atom A$,
the connections of both ends in both version A and B are included in $\atom A$.

Now, let a $\atom A \subset \atom X$ be atomical.
Let $S$ be the version A of the ZZ space ($\atom X A$ is version B), 
and 
let 
$F$ be a set of semantically interesting functions of $S$,
such as $f(S) = \mbox{``headcell of cell $c$''}$.


Atomical, nonintersecting changes
are easily seen to commute: $\atom A \atom B S = \atom B \atom A S$.

We want to define what it means for an atomical set to be independent.
There are several possibilities. 
First, we may say that if a semantically interesting property $f$
is preserved between versions, then applying $\atom A$ to $S$ 
shall also preserve this property.
$$
I_0(\atom A) =
    \forall f \in F \quad . \quad
		f(S) = f(\atom X S) \Rightarrow 
		f(S) = f(\atom A S)
$$
A stronger way to define independence is to say
$$
I_1(\atom A) =
    \forall f \in F \quad . \quad
		f(S) = f(\atom X S) \Rightarrow 
		    \forall \atom G \subseteq ( \atom X \backslash \atom A)
			f(\atom G S) = f(\atom A \atom G S)
$$
Another possibility for non-boolean properties is to 
say that a change is performed completely or not at all:
$$
I_2(\atom A) = 
    \forall f \in F \quad . \quad
	f(\atom A S) = f(S) \vee
	f(\atom A S) = f(\atom X S)
$$
Choosing between these possibilities is not easy.

Now, the remaining task is to divide $\atom X$ into 
disjoint independent atomical sets.
Their exact properties seem to depend on the types of functions.


\pagebreak

% Page 2: conflicts

Finding conflicts between different molecules is only slightly
more complicated.
Here, we take molecules (the groups of dependent atoms) for
$\atom A$, $\atom B$ etc.

Define the predicate $C_f(\atom B, \atom C, S)$
to mean that the two atoms $\atom B$ and $\atom C$ conflict
under function $f$ and space $S$.
We define this predicate as
$$
    C_f(\atom B, \atom C, S) =
    (
	((f(S) = f(\atom B S)) \vee (f(S) = f(\atom C S)))
	    \not \Rightarrow 
	    (f(S) = f(\atom B \circ \atom C S))
    )
$$

The interesting question is whether
$$
    ( C(\atom A, \atom B, S) \vee C(\atom A, \atom C, S) )
	= C(\atom A, \atom B \circ \atom C, S)
$$
for all molecules $\atom A$, $\atom B$ and $\atom C$.
If that is true, then the problem is solved.
Can this be proved for various groups of

\pagebreak

% Page 3: headcell

Now, consider the function set $H = \{ h(S, c) \}$ 
meaning the headcells of all cells
on the dimension under consideration.

The headcell of a cell can only be changed by a change
occurring in the same rank.
For $I_0$, we only need to consider the set of all cells in 
the space whose headcells do not change, and additionally only
the changes that touch the same rank. 

\pagebreak

% Page 4: before-after

Consider the relation $r(c, d)$, which is $-1$ if $c$ preceeds
cell $d$ on the given dimension, $1$ in the opposite case
and $0$ if the cells are on different ranks.


\end{document}
