\newif\ifdraft
\ifx\dontdraft\relax
\message{IFX TRUE}
\draftfalse
\else
\message{IFX FALSE}
\drafttrue
\fi

\def\fig#1{Fig.~\ref{#1}}

\draftfalse
% Final version: use times!
% \documentstyle[times,graphicx,ht01e]{article}
\documentclass[twocolumn]{IV02}
\usepackage{graphicx}
\usepackage{times}
% \draftfalse

\ifdraft
\renewcommand{\baselinestretch}{0.9}
\def\margincomment#1{
\marginpar{
\vbox{
\scriptsize #1
}
}}
\def\nakki#1{\marginpar{\bf Nakki: #1}}
\pagestyle{plain}
\else
\def\margincomment#1{}
\def\nakki#1{}

% Make the times fonts defaults again if not draft.
% \renewcommand{\sfdefault}{phv}
% \renewcommand{\rmdefault}{ptm}
% \renewcommand{\ttdefault}{pcr}

\fi


% Define a macro for your own margin notes here
% If it is ABSOLUTELY clear that something is a typo, go ahead
% and fix it. But don't make any major changes to the text itself
% yet. Also, only propose rearrangements in the margin: CVS doesn't
% work too well with rearrangements and edits done by several
% people. I will be the editor until wed or thu.
\def\tjl#1{\margincomment{Tjl: #1}}
\def\vp#1{\margincomment{vp: #1}}
\def\rr#1{\margincomment{rr: #1}}
\def\cat#1{\margincomment{cat: #1}}
\def\ajk#1{\margincomment{ajk: #1}}
\def\jvk#1{\margincomment{jvk: #1}}
\def\benja#1{\margincomment{benja: #1}}



% \documentstyle[times,graphicx,uist,isolatin1]{article}
\begin{document}

\newcommand{\url}[1]{{\fontsize{9pt}{8pt}\fontfamily{phv}\fontseries{c}\selectfont{#1}}}
\newcommand{\hyp}{\discretionary{}{}{}}


\bibliographystyle{plain}
\title{
Node Shaders: Visualizing Identity using Procedural Textures
}

\author{
\parbox[t]{8.9cm}{\centering
             {Tuomas J. Lukka\ \ \ and\ \ \ Janne Veli Kujala}\\
  Hyperstructure Group\\
  Dept.~of Mathematical Information Technology\\
  University of Jyv\"askyl\"a, PO.~Box~35\\
  FIN-40351~Jyv\"askyl\"a\\
  Finland\\
  {\tt lukka@iki.fi},\ \ {\tt jvk@iki.fi}}
  }

\maketitle

\ifdraft
\textwidth 6.6cm
\columnwidth 8.6cm
\onecolumn
% \textwidth 8.5cm
\marginparwidth 8.5cm
\fi

% \textwidth 3.5in
% \columnwidth 3.5in
\def\gzz{GZZ}

\begin{abstract}
Maintaining user orientation in 
in Focus+Context views of complicated graph-like structures
is challenging.  
As the fraction of nodes that invisible at each focus grows,
getting lost in the structure becomes easier.
In multi-linked (non-tree) graphs,
recognizing a previously seen location is vital.

We use procedural textures to give the user visual cues about
the {\em identity} of the nodes by 
making the nodes look different from each other randomly.
We discuss differences between node shaders
and conventional e.g.~Renderman shaders: a greater variability
is desirable, there is a need for a new type of 
noise function which is quite rarely nonzero etc.

We discuss an efficient implementation using batch rendering of the
cells and the unextended OpenGL 1.2 API, as well as an implementation
using a texture shader extension to OpenGL.

There are reasons why this technique is especially well suited for \gzz,
but it is generally applicable as well.

(manuscript for \url{http://www.graphicslink.demon.co.uk/IV02/})
\end{abstract}


% \tolerance=400 
\tolerance=500 
  % makes some lines with lots of white space, but      
  % tends to prevent words from sticking out in the margin

\def\vob{{\tt Vob}}
\def\vobscene{{\tt VobScene}}
\def\view{{\tt View}}
\def\key{{\tt key}}
\def\diff{{\tt diff}}

\def\unfin{\tiny}
\def\fin{\normalsize}


\section{Introduction}

Viewpoints:
\begin{itemize}
\item Identity in physical world; 
    for computer, two objects different even with same name, but
    visually {\em pixel-for-pixel} same.
    Also, the text is the only difference
\item Focus+Context / other Navigation: recognizing where I am
\item Imbuing different objects with a "feel" --- hard to quantify
\item Applying procedural textures used in motion pictures to visualization
\item Visualizing scalar discrete unordered data.
\item Visualization: providing visual cues --- here, we provide a LOT
\end{itemize}

Recognizing different people or animals or houses in real life
is possible since
the human eye can detect the minute variations in texture and shape
and use these cues to assign {\em identity} to different
objects of the same kind ("Janie has a white spot on her left front paw").

In this article, we apply procedural textures to artificially
create such differences in visualizations. 
The differences are used to give users 
cues when navigating
Focus+Context views where only a part of the information
is seen clearly at a time.

% During the last years, 
% Focus+Context views\cite{XXX} have been steadily gaining
% popularity.
% 
% 
% Focus+Context views often first embed the structure
% to be visualized onto either Euclidean or hyperbolic space and then
% display that space, selectively magnifying region around the current 
% focus\cite{fc-taxonomy}.
% 
% 
% In this article, we are concerned with Focus+Context views of
% graph-like data where the graph is {\em not} first embedded into
% any other space, be it Euclidean or hyperbolic.
% 
% These views focus on one (or more) nodes and generally display those nodes
% more prominently, and also display several layers of neighbouring nodes.
%
%When navigating such a view, it is vital to be able to orient
%oneself quickly when one returns to a location that has been seen
%before.
%The view should somehow allow the user to quickly perceive the {\em identity}
%of the nodes the user is looking at.
%
%
%On WWW, different sites have different looks, consisting of
%images, colors, fonts and layout. 
%
%It is not desirable to have to graphically design each node separately.
%
%Specifically, our primary research interest is the \gzz prototype
%which implements the ZigZag (ZZ) hyperstructure.
%
%The ZZ structure does not share some of the problems of general graphs
%that are attacked through embedding...

% -- confrontational :(
%While this type of view has been called {\em ad hoc} by some of the
%researchers in
%



%The use of the pre-attentive processes of human vision
%for visualization purposes has been the subject of much
%research recently. XXX in XXX...
%
%
%jhhjjjjjjhjhjjjuji[]
%]\
%
%
%,,,.mh t n jk .9       o                                              sssssssssssssssssss
%
%IDENTITY
%
%In this article, 

\section{Procedural textures}

% Early computer-generated images had an artificial, plastic feeling to them.
% This was because the surfaces in the images did not have any 
% of the imperfections that real surfaces do.

Procedural textures 
\cite{texturing-and-modeling,renderman-spec,advanced-renderman}
have been long used for photorealistic computer 
graphics. 
The \emph{Perlin noise function} 
% was introduced
% in the seminal work article\cite{perlin-noise-intro}.
is a repeatable band-limited random function that is fast to calculate.

Using noise functions,
it is simple to generate different instances of similar objects.
For example, consider drawing an apple tree. If one apple is created by
modeling it with a polygonal modeler and painting a texture on it, 
then either every apple will have to be done separately (too much
work) or all the apples will look exactly the same (unacceptable).
If the first apple is created as a procedural texture in an intelligent
way, then new, different yet similar apples may be generated simply
by using the same procedural texture with different 
set of random numbers.


\section{Node shaders}

Texturing is important\cite{haeberli93texture}.

In early prototypes of GZigZag, all cells looked exactly the same,
except for text. Since text is not pre-attentively processed,
all cells basically looked the same, which could be disorienting when moving
around a paradoxically connected hyperstructure.

The human eye is not used to having separate objects look exactly the same; 
natural objects always have slightly different shapes, scratches, different
colors or other minor distinguishing features.

Adding permanent imperfections and differences to graphical representations
of the cells can reduce the disorientation, since recognizing
cells at a glance (pre-attentively) would be easier.

Procedural textures may be used to make different cells look somewhat
different (but not too different, to avoid making the view a confusing jumble)
from each other.

One difference between the  photorealistic computer graphics 

In photorealistic computer graphics, when shading a surface, some control
over its appearance is usually desired. 

We tolerate outliers much more...

ALIASING NOT SO MUCH OF A PROBLEM

One possibility, as discussed by XXX in \cite{texturing-and-modeling}
is to use the usual Perlin noise and simply map the noise values so that
the desired ...

However, this becomes difficult when the ratio of spots...

WORLEY NOISE -> rare, different spots

Can have effects that only materialize in only every 100th node.
\cite{more-is-different}


\begin{figure*}
\centering
\includegraphics[width=\textwidth]{rects.eps}
\caption{
\label{fig-proctext}
Using procedural textures to add distinguishing imperfections to 
rectangles. 
All the rectangles shown were generated by a single procedural
texture. 
}
\end{figure*}

\section{A meta-noise function}

-- Which parameters can be tuned: tuning frequency of noise is difficult:
  the magnitude of the location parameter gets mixed in...
$$
    { \partial f(\lambda x) \over \partial \lambda } 
	= x f'(\lambda x)
$$
$$
    { \partial \over \partial \lambda }
\biggl(	\lambda f(\alpha x) + (1-\lambda) f(\beta x) \biggr)
	= f(\alpha x) + f(\beta x)
$$


\section{Technical details}

\label{sec-basicimpl}

Implementing the previous ideas at interactive speeds is not
as difficult as it might seem

When combining edgeless connections and procedural textures, the 
transition from one cell to another smooth. This can be achieved by 
blending the noise function at the shader position (which varies between
different cells) with the noise function at the object position 
(which is constant between different cells).

When designing such node shaders, an important choice is at what level
to perform the blending. For example, if the basic procedural texture
for the node is three combined functions,
$$
f(g(h(x))) \rightarrow \left\{
\begin{array}{l}
f(g(m(h(x), h(y), \alpha))) \\
f(m(g(h(x)), g(h(y)), \alpha)) \\
m(f(g(h(x))), f(g(h(y))), \alpha) 
\end{array}
\right.
$$
This choice depends on the nature of the functions in question
and is beyond the scope of this article.

XXX: 3D-TEXTURES

\begin{figure*}
\fbox{\vbox{\vskip 4in}}
\caption{
\label{fig-proctext-argh}
Interactive rendering of node shaders. 
Rendered with an NVIDIA GeForce3 Ti200 at XXX fps using 
the OpenGL NV\_texture\_shaders extension.
A Focus+Context view of a ZigZag structure using Gzz 
shaded with node shaders. The same structure is shown from two different
Foci.
}
\end{figure*}

\begin{figure*}
\centering
\includegraphics[scale=0.5]{proctext-nvshading.eps}
\caption{
}
\label{fig-proctext-nvshading}
\end{figure*}

Rubberiness, stretching...


For a static diagram, it might make sense to 

A NURBS surface is used for stretching the edgeless connection if
it is not properly aligned.

Bump mapping!

New consumer-level 
graphics cards such as the ATI Radeon and NVIDIA GeForce3 and GeForce4
support procedural vertex and texture manipulation to various levels. 
\cite{proudfoot01realtime}
The preliminary drafts of the OpenGL 2.0 specification (ref) are also
a great leap in this direction.
It seems only a matter of time until both techniques described
here can be implemented directly on the graphics accelerators using
vertex and pixel shaders and the optimizations of Section~\ref{sec-basicimpl}
will no longer be necessary.

\section{Rationale}

Instead of providing few cues consciously, such as fog or depth-of-field,
we provide a huge number of different cues randomly from which the user
can unconsciously select the best ones.

As discussed in \cite{hopfield01olfaction}, having several sensory inputs
makes it simple to tell apart different smells; we hope to use the same effect
on a slightly higher level in the visual cortex.

\section{Obtaining the software}

GZigZag is Free Software, distributed under the LGPL license of the
Free Software Foundation,
and can be obtained at 
\url{http://gzigzag.org}
and \url{http://gzigzag.sourceforge.net}. 
The features described here will be included in release 0.8.0 and can
be accessed in the public CVS repository before that.


\section{Conclusion and further work}

In this article, we have introduced a new (to the author's 
knowledge) technique for
visualizing data with a discrete hyperstructure: 
node-based procedural textures. 


Unsolved issues:
\begin{itemize}
\item How to use procedural textures consistently when the size and aspect
ratio of a node changes over time? In GZigZag, the user can add text to a node,
but it would be good if the node remained recognizable.
   \benja{
      Use tiles and show more of them as the cell grows. (You should be
      able to compute the different tiles by slightly modifying the
      random number seed when moving one tile to the left/right/top/bottom?)
      This handles aspect ratio changes that grow thecell, too. Changes
      that shrink the cell canbe handled by simply showing less of what's
      already there.
   }
\item How to make smoothly scalable procedural textures for use as OpenGL
textures? Building the final result from pieces would be good but what types
of pieces?
\end{itemize}



The graphical aspect of designing suitable procedural textures for use
as node backgrounds is beyond the scope of this article; the backgrounds
should be clearly distinct from each other but not draw attention
away from the text or make reading the text difficult.

The node-based procedural textures would be immediately applicable
e.g.~to file managers, where usually all files of the same type look
exactly the same, except for the names. 

We have not yet been able to research the quantitative benefits of
the techniques introduced here; we hope to be able to present this
at a later date.

% \bibliographystyle{abbrv}
\bibliography{gzigzag}

\end{document}
