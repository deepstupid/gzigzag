\documentclass{article}
% \usepackage{times}

\parindent=0pt
\parskip=10pt

\begin{document}
% \font\norm=ptmr at 10pt

% \norm

%\textwidth=2cm
\rightskip=5cm

\newcommand{\subject}[2]{%
\begin{list}{}{%
   \bf
   \settowidth{\labelwidth}{#1}%
   \setlength{\itemindent}{0em}%
   \setlength{\labelsep}{1em}%
   \setlength{\leftmargin}{\labelwidth}%
   \addtolength{\leftmargin}{\labelsep}%
   \setlength{\rightmargin}{\rightskip}%
   \setlength{\topsep}{0em}%
}%
\item[{#1}] #2%
\end{list}%
}

\def\pti{
4.3.2. Algorithm
}

\def\ptii{\rightskip=-5.0cm
The actual algorithm used by the name server will depend on the local OS
and data structures used to store RRs.  The following algorithm assumes
that the RRs are organized in several tree structures, one for each
zone, and another for the cache:
}

\def\ptiii{\rightskip=-5.0cm
   1. Set or clear the value of recursion available in the response
      depending on whether the name server is willing to provide
      recursive service.  If recursive service is available and
      requested via the RD bit in the query, go to step 5,
      otherwise step 2.
}

\def\ptiiii{\rightskip=-2.5cm
   2. Search the available zones for the zone which is the nearest
      ancestor to QNAME.  If such a zone is found, go to step 3,
      otherwise step 4.
}

\def\ptv{\rightskip=2.0cm
   3. Start matching down, label by label, in the zone.  The
      matching process can terminate several ways:
}

\def\ptvi{\rightskip=-1.0cm
         a. If the whole of QNAME is matched, we have found the
            node.

            If the data at the node is a CNAME, and QTYPE doesn't
            match CNAME, copy the CNAME RR into the answer section
            of the response, change QNAME to the canonical name in
            the CNAME RR, and go back to step 1.

            Otherwise, copy all RRs which match QTYPE into the
            answer section and go to step 6.
}

\def\ptvii{\rightskip=-1.0cm
         b. If a match would take us out of the authoritative data,
            we have a referral.  This happens when we encounter a
            node with NS RRs marking cuts along the bottom of a
            zone.

            Copy the NS RRs for the subzone into the authority
            section of the reply.  Put whatever addresses are
            available into the additional section, using glue RRs
            if the addresses are not available from authoritative
            data or the cache.  Go to step 4.
}

\def\ptviii{
         c. If at some label, a match is impossible (i.e., the
            corresponding label does not exist), look to see if a
            the "*" label exists.

            If the "*" label does not exist, check whether the name
            we are looking for is the original QNAME in the query
            or a name we have followed due to a CNAME.  If the name
            is original, set an authoritative name error in the
            response and exit.  Otherwise just exit.

            If the "*" label does exist, match RRs at that node
            against QTYPE.  If any match, copy them into the answer
            section, but set the owner of the RR to be QNAME, and
            not the node with the "*" label.  Go to step 6.
}

\def\ptviiii{\rightskip=1.0cm
   4. Start matching down in the cache.  If QNAME is found in the
      cache, copy all RRs attached to it that match QTYPE into the
      answer section.  If there was no delegation from
      authoritative data, look for the best one from the cache, and
      put it in the authority section.  Go to step 6.
}

\def\ptx{\rightskip=0.0cm
   5. Using the local resolver or a copy of its algorithm (see
      resolver section of this memo) to answer the query.  Store
      the results, including any intermediate CNAMEs, in the answer
      section of the response.
}

\def\ptxi{\rightskip=-2.0cm
   6. Using local data only, attempt to add other RRs which may be
      useful to the additional section of the query.  Exit.
}


\def\terminology{
{
\subject{\underline{3.1.}}{Name space specifications and terminology\hfill
	 [\underline{RFC1034}]}

The domain name space is a tree structure.  Each node and leaf on the
tree corresponds to a resource set (which may be empty).  The domain
system makes no distinctions between the uses of the interior nodes and
leaves, and this memo uses the term "node" to refer to both.

Each node has a label, which is zero to 63 octets in length.  Brother
nodes may not have the same label, although the same label can be used
for nodes which are not brothers.  One label is reserved, and that is
the null (i.e., zero length) label used for the root.

The domain name of a node is the list of the labels on the path from the
node to the root of the tree.
}
}

\def\typedef{
{
\subject{\underline{4.3.3.}}{Wildcards\hfill [\underline{RFC1034}]}

\ldots

Wildcard RRs can be thought of as instructions for synthesizing RRs.
When the appropriate conditions are met, the name server creates RRs
with an owner name equal to the query name and contents taken from the
wildcard RRs.

\ldots

The contents of the wildcard RRs follows the usual rules and formats for
RRs.  The wildcards in the zone have an owner name that controls the
query names they will match.  The owner name of the wildcard RRs is of
the form ``*.\textless anydomain\textgreater'', where
\textless anydomain\textgreater is any domain name.
\textless anydomain\textgreater should not contain other * labels, and should
be in the authoritative data of the zone.

\ldots

    X.COM           MX      10      A.X.COM

    *.X.COM         MX      10      A.X.COM

    A.X.COM         A       1.2.3.4
    A.X.COM         MX      10      A.X.COM

    *.A.X.COM       MX      10      A.X.COM

This would cause any MX query for any domain name ending in X.COM to
return an MX RR pointing at A.X.COM.  Two wildcard RRs are required
since the effect of the wildcard at *.X.COM is inhibited in the A.X.COM
subtree by the explicit data for A.X.COM.  Note also that the explicit
MX data at X.COM and A.X.COM is required, and that none of the RRs above
would match a query name of XX.COM.

}
}

\def\RFCrefvi {\rightskip=4.0cm
\subject{\underline{2.3.5.}}{Special Considerations with CNAME RRs\hfill
	 [\underline{RFC2535}]}
   There is a problem when security related RRs with the same owner name
   as a CNAME RR are retrieved from a non-security-aware server. In
   particular, an initial retrieval for the CNAME or any other type may
   not retrieve any associated SIG, KEY, or NXT RR. For retrieved types
   other than CNAME, it will retrieve that type at the target name of
   the CNAME (or chain of CNAMEs) and will also return the CNAME.  In
   particular, a specific retrieval for type SIG will not get the SIG,
   if any, at the original CNAME domain name but rather a SIG at the
   target name.
\if 0
   Security aware servers must be used to securely CNAME in DNS.
   Security aware servers MUST (1) allow KEY, SIG, and NXT RRs along
   with CNAME RRs, (2) suppress CNAME processing on retrieval of these
   types as well as on retrieval of the type CNAME, and (3)
   automatically return SIG RRs authenticating the CNAME or CNAMEs
   encountered in resolving a query.  This is a change from the previous
   DNS standard [RFCs 1034/1035] which
   prohibited any other RR type at a node where a CNAME RR was present.
\fi
}

\def\RFCrefviii {
\subject{\underline{3.}}{Negative Answers from Authoritative Servers\hfill
	 [\underline{RFC2308}]}

   Name servers authoritative for a zone MUST include the SOA record of
   the zone in the authority section of the response when reporting an
   NXDOMAIN or indicating that no data of the requested type exists.
   This is required so that the response may be cached.  The TTL of this
   record is set from the minimum of the MINIMUM field of the SOA record
   and the TTL of the SOA itself, and indicates how long a resolver may
   cache the negative answer.  The TTL SIG record associated with the
   SOA record should also be trimmed in line with the SOA's TTL.

   If the containing zone is signed [\underline{RFC2065}] the SOA and
   appropriate NXT and SIG records MUST be added.
}
\def\RFCrefviiii {\rightskip=5.0cm
\subject{\underline{6.}}{Negative answers from the cache\hfill
	 [\underline{RFC2308}]}

   When a server, in answering a query, encounters a cached negative
   response it MUST add the cached SOA record to the authority section
   of the response with the TTL decremented by the amount of time it was
   stored in the cache.  This allows the NXDOMAIN / NODATA response to
   time out correctly.

   If a NXT record was cached along with SOA record it MUST be added to
   the authority section.  If a SIG record was cached along with a NXT
   record it SHOULD be added to the authority section.
\if 0
   As with all answers coming from the cache, negative answers SHOULD
   have an implicit referral built into the answer.  This enables the
   resolver to locate an authoritative source.  An implicit referral is
   characterised by NS records in the authority section referring the
   resolver towards a authoritative source.  NXDOMAIN types 1 and 4
   responses contain implicit referrals as does NODATA type 1 response.
\fi
}
\def\RFCrefx {\rightskip=1.0cm
\subject{\underline{5.4.}}{Receiving RRSets\hfill
	 [\underline{RFC2181}]}
   Servers must never merge RRs from a response with RRs in their cache
   to form an RRSet.  If a response contains data that would form an
   RRSet with data in a server's cache the server must either ignore the
   RRs in the response, or discard the entire RRSet currently in the
   cache, as appropriate.  Consequently the issue of TTLs varying
   between the cache
 and a response does not cause concern, one will be
   ignored.  That is, one of the data sets is always incorrect if the
   data from an answer differs from the data in the cache.  The
   challenge for the server is to determine which of the data sets is
   correct, if one is, and retain that, while ignoring the other.  Note
   that if a server receives an answer containing an RRSet that is
   identical to that in its cache, with the possible exception of the
   TTL value, it may, optionally, update the TTL in its cache with the
   TTL of the received answer.  It should do this if the received answer
   would be considered more authoritative (as discussed in the next
   section) than the previously cached answer.
}
