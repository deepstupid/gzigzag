% This paper is aiming to be an introduction to ZigZag technology.

\documentclass[a4paper]{article}

\usepackage[latin1]{inputenc}

\usepackage{beton}

\ifx\dontdraft\undefined
%Draft
\newcommand{\marginaali}[1]{\marginpar{#1}}

\setlength{\marginparwidth}{3cm}
\setlength{\textwidth}{13cm}
\setlength{\oddsidemargin}{1.3cm}
\else
%Non-draft
\newcommand{\marginaali}[1]{}
\linespread{1.6}
\fi

\newcommand{\nakki}[1]{\marginaali{\textbf{\small Nakki: #1}}}
\newcommand{\ajk}[1]{\marginaali{\small ajk: #1}}
\newcommand{\tjl}[1]{\marginaali{\small tjl: #1}}
\newcommand{\benja}[1]{\marginaali{\small Benja: #1}}

\begin{document}

\title{GZigZag technology}

\author{(ordering?) AJK, RR, TJL, THN\dots}

\maketitle

\begin{abstract}
  \nakki{Deadline: 2001-03-31.  Abstract needs to be about 300 words.}
We introduce Ted Nelson's ZigZag invention 
from the point of view of the programmer.
ZigZag is a hyperstructure which consists of
cells and dimensions that connect them bidirectionally.
As a platform for software
development, ZigZag provides
persistent data structure, efficient network transparency and
a global system for interconnecting data. 
Its difference from conventional approaches such as object-orientation is
discussed.

ZigZag also allows simple development of flexible non-textual programming languages
\tjl{Should mention same overall structure and framework for different
        languages!}
that emphasize the program's structure over its representation.  These
languages are similar but not identical to visual programming languages.

The visual platform provided by our implementation, GZigZag is
also introduced. The visual system allows modular addition of new views
and provides a framework for animation between different views of
the same cells.
\end{abstract}

\bigskip

\begin{verbatim}
$Revision: 1.18 $  
$Date: 2001/03/28 06:32:00 $
$Author: ajk $
\end{verbatim}

\newpage

\section{Introduction}

The basic idea of ZigZag, from one perspective, is
to represent all the (persistent/state/...? Not register and 
small-level stuff, optimization et.) information in a computer
in a generic structure model.

Instead of building all structures using the RAM and pointer model,
structures are instead built using nodes (cells) and 
two-directional connections between them.

The two-directional connection between the cells is a primitive
in this system.

Relational databases...

\begin{itemize}
\item Structure axioms! (a formal description of the structure)
        \item Relationship to relational databases' model
        \item unique ID for just about everything
        \item Xanadu text/media model in cells
        \item versioning
          \begin{quotation}
            \emph{From Benja}: Versioning in zz has a complex triple
            nature. We imagine people using their zz spaces as 'the
            system'. Such a system space must be linear in its
            microversions-- no branching!  That's because having
            different 'leading edge' versions of the system don't make
            sense-- kinda like the users would have to choose at
            startup 'which version of the system would you like to
            use?' But for 'documents', branching does make sense. In
            classical computer worlds, document = file, i.e. is well
            defined; that makes the task of versioning easier (ref.
            Ted's OSMIC)-- the document as a whole has versions, not
            its parts. In ZZ, that's more complex b/c you want to be
            able to version arbitrary collections of cells. So, that
            level, including how to implement branching of parts of
            the space when the space as a whole can't branch, is the
            second. It's the undo/redo level, while the first level is
            just kinda lookin' back in time. The third level is above
            microversions-- which everything before is about-- and
            about 'user's versions', i.e. versions of a document or
            part of it which are important for a user; but this
            doesn't necessarily belong here, as its a structure and UI
            issue, not one of the underlying system (although to
            explain the usability of versioning, this should be hinted
            at).
                
                 Build-in \emph{branching} versioning in the system is AFAIK
                 something nobody has done so far (linear undo/redo is mostly
                 build by the application programmers individually, even).
                 \ajk{CVS does it.}
                 \benja{Right. I haven't seen any applications using CVS as
                        a storage system, though. Are there any? But well,
                        you're right, one could build them even if there
                        aren't.}
             \end{quotation}
             TJL: Not really: I saw in the Handbook of Theoretical
             Computer Science references to "persistent data
             structures" in a different way from the one usually meant
             - meaning exactly what we mean by versioning here.  It
             would be worth taking a look at those papers for 1) the
             algorithms, 2) references for this paper.
             \ajk{\cite{persistent-ds}} Also, Ted's OSMIC/INLUV. GUI's
             (like Swing) action encapsulation.

                Benja: The algorithms aren't directly usable because they're
                designed for a node/pointer structure. ZZ connections are
                "two-dimensional": cell plus dimension. Artifically mapping
                this to a pointered structure would probably not save us any,
                even if the persistency algorithms we use are very effective
                for pointered structures. Well, I'll play that through. But
                I suspect that if anything, we should use a modified version
                of their ideas.
             
             AJK: The article~\cite{persistent-ds} talks about
             ephemeral data structures (structures where only the
             newest version is accessible at all), partially
             persistent DSs (all versions are accessible but only only
             one is mutable) and fully persistent DSs (all versions
             are both accessible and mutable).  IMHO we should be
             aiming for full persistence if we can do it without
             sacrificing too much cycles.
             
             Benja's concern about users not needing to choose
             different versions of the system is IMHO a user interface
             issue.  One can use only one branch even if branching is
             allowed.
             
             I think it would be possible and even worthwhile to
             implement fully persistent documents on top of a fully
             persistent space, assuming that the cost of branching is
             proportional to the number of cells and connections
             actually changing and not proportional to the size of the
             space.  We could then easily map document branches to
             space branches.
                
                Benja: A user interface issue? Well, yes. It's a UI issue if it
                is undefined which configuration the computer uses at startup. 
                But in my humble opinion, that doesn't make it less important 
                to the system design... ;)
                
                Or do you want to say if one wants to use different branches,
                one would have to choose between them at startup? I.e., one
                could not use branching documents if one doesn't want to
                choose the "current space" at startup?
                
                I think that the implications of branching the space to branch
                a document are too nasty. When you view document D from
                timestamp i, and there are two consecutive versions at
                timestamps j and k, how do you find out whether these contain
                branches of D? The change that lead to j may not have changed
                anything inside the document, but some change a lot deeper than
                j may have affected D. So somewhere along j there would be a
                branch of D. These kind of lookups would be easier if the
                data structure was designed for being efficient with
                branching documents, not branching spaces.
                
                I don't know if I'm talking clearly here. I'll write a school
                paper about versioning issues in ZZ sometime soon, so then
                I'll take the time to clarify.
                
                One thing is that I still don't understand why you would want
                to branch the whole space. What for? Why do you need space
                branches so often that you intend to make that the efficient
                case, not document branches? Because of course one could at
                least as easily map space branches to document branches as
                the other way around, with the slight difference that these
                space branches would have only branches of the home cell, not
                the home cell itself, and thus wouldn't be "the current space"
                (although technically, they'd be part of the current space,
                but there wouldn't be a single connecting path from the
                home cell to the branched spaces' home cells).
                
                (I find the "persistent" terminology not useful. I'd prefer
                unversioned, linearly versioned, fully versioned.)
                
                AJK: I would encode the document name into the branch
                that was made for that document.  And when we are
                talking about branching versions, version tags will
                not be simple timestamps any longer.
                                
        \item persistence
        \item separation of view and data
        \begin{itemize}
                \item Usually APIs exist but not easily publicly accessible
                        in programs. MS's COM is one example, but
                        how often does it really provide true 
                        reusability? It allows embedding full components
                        but for more fine-grained access, there's
                        just about nothing.
                \item structure vs. API
                \item easy to define new views
        \end{itemize}
        \item Clangs -- non-textual and non-visual languages (structure programming:-)
          (need to compare to visual programming, eg.~Prograph)
          \ajk{Note: Microsoft's Visual Basic etc.\ are \emph{not} visual programming languages}
        \item merging?
        \item hyper-orthogonality
        \item applitudes -- examples
        \item focus+context views
\end{itemize}

\section{A formal description of the GZigZag structure}



\section{Conclusion}

\bibliographystyle{abbrv}
\bibliography{gzigzag}


\end{document}

