\documentclass[twocolumn]{article}
\usepackage[boldsans]{concmath}
\usepackage[latin1]{inputenc}
\usepackage[T1]{fontenc}

\newcommand{\ra}{$\rightarrow$}
\newcommand{\nakki}[1]{\marginpar{\textbf{\small Nakki: #1}}}

\textwidth=13cm
\columnsep=1cm

\begin{document}
\title{A focus+context view for threaded message forums using 
        the GZigZag hyperstructure toolkit}
\author{Hyperstructure group\\ (ketk� kaikki haluaa olla mukana? Who
        want to be in on this?)}
\maketitle

\marginparwidth=3cm
\oddsidemargin=1.3cm

\begin{verbatim}
$Revision: 1.8 $  
$Date: 2001/02/28 10:34:50 $
$Author: tjl $
\end{verbatim}

\section{Introduction}

	\begin{itemize}
		\item General discussion
		\item References to focus+context
		\item discussion of hyperstructure and zz.
		\item Message forums - references
		\item message forum / chat visualizations: Vesa said
			he had something about chat f+c.
		\item (or is this another paper:) references to
			timing, frame rate, response times etc.
			What feels responsive, what jammed.
	\end{itemize}

\section{The ZigZag structure}

                \begin{itemize}
		\item Cells and dimensions
		\item Speciality: just about everything has an unique
			identity in the same space (cells).
		\item A new way of representing state of program - a new,
			associative memory model
                \end{itemize}

\section{The GZigZag platform}

	\begin{itemize}
	\item An experimental prototype implementation of the structure
	\item ALL program state stored in a persistent ZZ structure, including
		cursors and view parameters
	\end{itemize}

	Viewing:

                \begin{itemize}
                \item Cell id \ra VOb id \ra automatic animation
                \item separate cellviews and views
                \item new views easy to write: just slap VObs on screen .
			general animation code works and handles the rest.
                \end{itemize}

        \subsection{Persistence}
		??

        \subsection{Views}
		??


        \subsection{Support for focus+context}

                \begin{itemize}
		\item How easy it is to make f+c views in zz, 
		\item code is given accursed cell, traverses neighbours
			and slaps them on screen
		\item animation autosupported, as well as clicking
		\item however, random placement of things more difficult:
			cursor is on cell. Some f+c paradigms require being
			able to place nodes anywhere by dragging.
		\item The planned metafont-like even simpler way
                \end{itemize}

\section{A focus+context view for message forums}
	
	\begin{itemize}
	\item Move among message tree with text of each (short) message
		seen.
	\item Long messages \ra have to have focus somewhere inside the message.
	\item authors --- more hyperstructure
	\item bookmarks, own ranks, --- even more hyperstructure 
	\item joystick control --- force feedback when focus moves
		inside message and then is pushed over the edge to the next
		message
	\end{itemize}

\section{A focus+context view for message forums}
	
	\begin{itemize}
	\item Nature of slashdot, discussions
	\item parsing data \ra how difficult?
	\item moderation, shown as e.g. message size
	\end{itemize}

\section{Problems to be solved}

	\begin{itemize}
	\item Rotated text is slow
	\end{itemize}


\section{Conclusion}


	\begin{itemize}
	\item Combined with collaborative filtering a la grouplense, etc.
	\item filtering by viewing - message size, prominence
	\item interesting system presented
	\item more refs to other work
	\item Jos saadaan t�m� sovellus slashdottiin ennen deadlinea, 
		viittaus sinne keskusteluun.
	\end{itemize}


\end{document}
