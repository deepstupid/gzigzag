\documentclass[twocolumn]{article}
\usepackage[boldsans]{concmath}
\usepackage[latin1]{inputenc}
\usepackage[T1]{fontenc}

\newcommand{\ra}{$\rightarrow$}
\newcommand{\nakki}[1]{\marginpar{\textbf{\small Nakki: #1}}}

\textwidth=13cm
\columnsep=1cm

\begin{document}
\title{Content linking and document visualization}
\author{Hyperstructure group\\ (ketk� kaikki haluaa olla mukana? Who
        want to be in on this?)}
\maketitle

\marginparwidth=3cm
\oddsidemargin=1.3cm

\begin{verbatim}
$Revision: 1.2 $  
$Date: 2001/03/02 16:00:49 $
$Author: tjl $
\end{verbatim}

\section{Introduction}

	\begin{itemize}
		\item l�pin��
	\end{itemize}

\section{The Xanadu content linkage model}

The Xanadu content linkage model\cite{XXX} is 
simple but powerful.
The WWW model of one-directional links
is essentially the simplest computerization
of conventional paper
citations, such as ``see page 64'' or ``see Smith et al, 1999''.






	\begin{itemize}
		\item XLink?
	\end{itemize}

\section{A prototype implementation on the GZigZag platform}

\section{An example set of interlinked documents: RFCs}

	\begin{itemize}
		\item omat kommentit
	\end{itemize}


\section{Conclusions}

\end{document}
