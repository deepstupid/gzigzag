\documentclass[a4paper]{article}


\usepackage[boldsans]{concmath}
\usepackage{euler}
\usepackage[latin1]{inputenc}

\usepackage{amsmath}

\ifx\dontdraft\undefined
%Draft
\newcommand{\marginaali}[1]{\marginpar{#1}}

\setlength{\marginparwidth}{3cm}
\setlength{\textwidth}{13cm}
\setlength{\oddsidemargin}{1.3cm}
\else
%Non-draft
\newcommand{\marginaali}[1]{}
\linespread{1.6}
\fi

\newcommand{\nakki}[1]{\marginaali{\textbf{\small Nakki: #1}}}
\newcommand{\ajk}[1]{\marginaali{\small ajk: #1}}
\newcommand{\benja}[1]{\marginaali{\small Benja: #1}}

\DeclareMathOperator{\pos}{pos}
\DeclareMathOperator{\out}{out}
\DeclareMathOperator{\outmid}{outmid}
\DeclareMathOperator{\outdepth}{outdepth}
\DeclareMathOperator{\nsib}{nsib}
\DeclareMathOperator{\child}{child}
\DeclareMathOperator{\fchild}{fchild}
\DeclareMathOperator{\lchild}{lchild}
\DeclareMathOperator{\mchild}{mchild}
\DeclareMathOperator{\mchildren}{mchildren}

\begin{document}

\title{Rasters(?) and constraints}

\author{Tjl}

\maketitle

\begin{abstract}
  XXX
\end{abstract}

\begin{verbatim}
$Revision: 1.23 $  
$Date: 2001/03/20 09:45:12 $
$Author: tjl $
\end{verbatim}


\section{Introduction}

What
metapost does that makes life easier for graphics programmers is actually
quite simple - it solves linear equations. 
The power here comes from the simple idea
in metapost that you specify the \emph{constraints} of the problem, not the solution.
There's much literature in the field of constraint-solving and constraint problems,
some of it graphical. 
The Tk (of Tcl/Tk) packer is one example of such a system
as well, quite succesful in its own space. So is \TeX's box and glue system.
\nakki{Literature!}

Another place for constraints would b

Ted Nelson's idea of rasters...


In ZigZag, all information is represented as an amorphous hyperconnected
network of cells. \nakki{Dimensions bla bla bla}
\nakki{Most views have been defined for data with simple schemas!}


The concept of a raster is taking a part of the structure that fits some 
simple schema
and leaving the rest out --- from the current view. This operation is
intrinsically tied with the focus+context nature of the views.

The raster is described as a graph with labeled arcs and nodes (XXX???
Trouble ahead...).
The labels and the graph obey some simple structural rules,
which makes handling them easier than just plain general ZZ structure.

One cell in the structure may correspond to more than one nodes
in the raster. This has no effect on the raster itself, as
the raster is seen as the main structure here.
Often rasters will have two variants based on this: one where
multiple instances are allowed and another where they are
forbidden and breadth-first search is used to see which instance
to keep.

Fitting the raster onto the screen is easily described using 
constraints.
For the constraints, we define general functions of the nodes and
pairs of the nodes to integers, reals or vectors and
describe the constraints between them.
The functions between pairs of nodes are generally from \emph{unordered}
pairs, for reasons we shall see below.

Constraints have recently been used in graphical applications
by Badros\cite{badrosthesis, badros99constraint, badros98cassowary},
see also references in these papers to earlier work.

\section{Rasters}

A raster is a set of nodes, one or more relations on the nodes
and zero or more mappings from the nodes to something else (e.g.~integers).

This definition is slightly vague, since the nature of rasters
can vary widely. 

Essentially, a raster is a structured set of nodes, where
the structure is not completely rigid.

\section{Common raster structures}

\subsection{The centered rank raster component}

The centered rank raster component (CRRC)
is usually too simple to be used as a raster 
\emph{per se}, but it is used as a component of several rasters.
It is simply a linear finite, semi-infinite or infinite
list of cells, with one cell being designated as the center.
Therefore, each connection has a \emph{direction} associated
with it: if we take the center as the origin, then all 
the connections are away from the origin in this sense.

It becomes easy to express thing in a declarative way if
we can address the set of all connections in different ways.
Let us first name the nodes by $r_i$ where $i$ is the index 
of the node on this rank. The symbol $R$ represents the whole
raster.
Then, we define the following
sets of pairs of indices:
\begin{align}
        \pos(R) &=  \bigl \{\, (r_i, r_j) \bigm| j = i+1 \, \bigr \} \\
        \out(R) &=  \bigl \{\, (r_i, r_j) \bigm| |j| = |i|+1 
                                \text{ and } |i-j| = 1 \, \bigr \}
\end{align}
so that  $\pos$ is the set of poswards links on the rank and $\out$ is the
set of links on the rank outwards from the center cell.

This structure is one of the reasons why using procedural code
to handle view is complex: in procedural code, the special
nature of the center node has to be accounted for explicitly,
as do the directions of the connections. This requires
duplication of the same loop or including a direction variable
in several places inside the loop.

An example view using this raster could be as follows.
The behaviour of having the size of the cells and the length
of the interval between them shrinking from the center
can be described by the constraints among the functions
$x(t)$, the location of the cell, $w(t)$, the width of the cell
and $s(p)$, the distance between a pair of cells.
\begin{align}
        \forall (r, s) \in \out(R) \;&:\;
                w(s) = C w(r) \\
        \forall (r, s) \in \out(R) \;&:\;
                s((r, s)) = D w(r) \\
        \forall (r, s) \in \pos(R) \;&:\;
                x(s) = x(r) + w(r) + s((r, s)) 
\end{align}
The procedural representation for the same layout is not as simple.
The notable idea is the transfer of the width $w$ outwards onto
the edge between $r$ and $s$ and then the use of this information
in the poswards-moving condition on the coordinates $x$.
This corresponds well to the 
conceptual structure and allows the simple expression of this structure.

The centered rank is usually simply a whole or a part of a rank
in the ZigZag structure but may also be something else.

The view in GZigZag that shows a single stream of text can be
seen as being based on this raster.

\subsection{The grill raster}

The grill raster consists of one \emph{backbone} CRRC and a 
\emph{rib}
CRRC for each element of the backbone. The element of the backbone
is the center element of the corresponding rib.
\nakki{Parempi nimi?}

The definitions here are similar, except that we define 
the sets
$\pos_b(R)$  and
$\pos_r(R)$ separately for the backbone and the rib (similarly
for out.


\subsection{The centered rank tree raster}

The CDTree raster is used e.g.~in GZigZag's Vanishing View. \nakki{Fig}
Here, the raster is tree-shaped in the same way as the grill raster,
but it is continuing.
Say we have a $d$-dimensional CDTree raster. 
Then each cell can be on $d$ different CRRCs, and it will be 
the center on all but one of them (except for the center cell
which is the center cell on all its ranks).
For each index of dimension $i$ we define
$\pos_i(R)$  and
$\out_i(R)$.


\subsection{The tree raster}

The tree raster is different from the above rasters since
it is not as directly connected with the ZigZag structure.
Therefore, the CRRC will not find as much use here.

The tree raster is somewhat different from a normal tree graph:
the children of a node are ordered. Also, the parent-child relationship
is separate from the ``center'' cell in the sense of the ZigZag cursor.

There are several different ways in which the visual
\nakki{improve. These relations suck.}
parent-child relations may be desirable to express. 
Because of this, we have to define more sets of relations for this
raster.
%
\begin{description}
\item[$\nsib(R)$] is the set of child --- next child relations.
\item[$\child(R)$] is parent --- child relation.
\item[$\fchild(R)$] is parent --- first child relation.
\item[$\lchild(R)$] is parent --- last child relation.
\item[$\mchild(i, R)$] is parent --- middle child relation; $i$ is
the index $0$, $1$ or $2$: if there are an odd number of children,
$0$ gives the middle one, otherwise $1$ and $2$ give the upper and
lower children.
\item[$\mchildren(R)$] gives a relation between
the upper middle and lower middle children.
\item[$\outmid(R)$] is analogous to $\out$ above: 
it is a relation from the middle children outwards.
\item[$\outdepth(R)$] gives the relation outwards
from the cursor in the strict tree sense: parent is one step away,
siblings two steps etc.
\end{description}



\section{Descriptions of specific rasters}

The specific mapping of a ZigZag structure onto a raster,
starting from a given cell must also be described.


\section{CellViews in constraints}

In GZigZag, cellviews are an important concept. They determine
how a particular node will be shown.

Integrating cellviews to the constraint framework is not quite trivial
since there are different variants of rasters where the width for a cell
comes from the outside or from the cellview.

Another is that cellviews may demand the implementation of a constraint
hierarchy: given a fractional size, a cellview may detemine an optimum
width and height, but also a relationship between width and height
when one of them is determined.

Also, cellviews determine the location of the "alignment point" inside
the cell, used for the location of the connections.

In the initial, simple model we shall only treat cellviews using the
simple constraints
\begin{align}
	w &= C_w s \\
	h &= C_h s \\
	x_a &= l_x(s, w, h) \\
	y_a &= l_y(s, w, h) 
\end{align}
where $s$ is the size fraction, and $l_x$ and $l_y$ are 
affine functions. The constants $C_w$ and $C_h$ as well as $l_x$ and $l_y$
are allowed to depend on the current node.

The Vanishing and StretchVanishing views of GZigZag differ essentially
on the part of $C_w$ and $C_h$, not the actual view.

The above model is too simple, though: most fonts' fontmetrics
do not shrink linearly with the font. the above constraints
can be replaced with the directed assignments
\begin{align}
	w &\leftarrow f_w(s) \\
	h &\leftarrow f_h(s) 
\end{align}
but this assumes that $s$ has been determined. Usually this is not a problem.

\section{Constraints}

Local-Neighbour-Init-Global !!!


\section{Practical implementation}

The systems of constraints described here are relatively 
different from the usual literature. In the literature,
it is usually assumed that the system of constraints is relatively
stable and new constraints may be added or old ones removed or changed
on a one-by-one basis.

In the raster to screen problem, the raster changes completely between
the frames so the whole system needs to be rebuilt. This requires different
types of optimizations and because of this, we define some restricted
classes of view constraints below that are possible to implement efficiently.

\subsection{Propagatable constraints}

Algorithms that create rasters are able to use less space if 
all nodes' variable assignments need not be memorized.
This is possible if the constraints are structured  so that 
the value of all variables of a node can be determined given
a particular set of nodes.

Formally, we have a set-valued function of nodes $f(t)$, and 
the given constraint program is propagatable under $f$, iff
all values for node $t$ can be exactly determined from 
the values for the nodes in $f(t)$.

This restriction does not allow constraint hierarchies and is
relatively restrictive but on the other hand, for carefully
chosen $f$ it is quite simple to implement.



\subsection{Two-pass propagatable constraints}


\section{Once again}

{\bf This section discusses the same topics as the previous ones
from a different POV. This whole manuscript will be completely
reorganized at a later point}

Constraints on a view built on top of a raster can be 
grouped into four distinct groups:
\begin{description}
\item[Local]
	Constraints between different functions' values on the
		same node $x$.
\item[Neighbour]
	Constraints between nodes $x$ and $y$ which are
		related by some raster relation.
\item[Init]
	Constraints on one or more nodes that define values
		of some functions initially.
\item[Global]
	Constraints that place non-local restrictions on the values.
\end{description}

The distinction between init and global constraints
has to do with the nature of propagation of constraints:
a constraint that places one end of a rank at a given point
and gives the cells a size is Init, but a constraints
that place two ends of a rank, implicitly defining the sizes
of the cells, are Global.

Local constraints are related to cellviews.

Except for Global constraints, which are usually related to the
Neighbour constraints, the different constraints can be good points
to split the constraint sets apart as modules. For instance,
the Local constraints can define whether the cells' sizes grow
with increasing amount of text.



% \begin{itemize}
% \end{itemize}

\section{Conclusion}

The concept of a raster doesn't make much sense in procedural languages.
However, in constraint-based XXX the concept acquires a new depth...




%$$
%\{ \{ {A\rightarrow
%{{\frac{-2\,{{\left( -1 \right) }^{-k - m}}\,\left( c - 1 \right) \,
%{{\left( -k + c\,\left( k - m - 1 \right)  + m -
%{\sqrt{4\,\left( c - 1 \right) \,k\,m +
%{{\left( k + m - c\,\left( k + m - 1 \right)  \right) }^2}}} \right)
%}^k}\,{{\left( k - m + c\,\left( -1 - k + m \right)  -
%{\sqrt{4\,\left( c - 1 \right) \,k\,m +
%{{\left( k + m - c\,\left( k + m - 1 \right)  \right) }^2}}} \right)
%}^m}\,{{\left( k + m - c\,\left( k + m - 1 \right)  +
%{\sqrt{4\,\left( c - 1 \right) \,k\,m +
%{{\left( k + m - c\,\left( k + m - 1 \right)  \right) }^2}}} \right)
%}^{-k - m}}}{2 - k - m + c\,\left(  k + m - 1 \right)  +
%{\sqrt{4\,\left(  c - 1 \right) \,k\,m +
%{{\left( k + m - c\,\left(  k + m - 1 \right)  \right) }^2}}}}}}}\} \}
%$$

\bibliographystyle{alpha}
\bibliography{gzigzag}

\end{document}

