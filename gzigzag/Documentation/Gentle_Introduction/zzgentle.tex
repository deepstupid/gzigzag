% Copyright(C) 1999 Tuomas J. Lukka
% Redistribution permitted
%
% XXX redo pics in gpic.
%
%
\documentclass[12pt]{article}


\usepackage{rcs}
\RCS $Date: 2000/09/13 14:55:57 $
\RCS $Revision: 1.17 $
\date{Rev.\RCSRevision~~\RCSDate}


\usepackage{umlaut}
\usepackage{times}
\usepackage{xy}
\xyoption{frame}
\xyoption{matrix}
\xyoption{arrow}
\xyoption{curve}
\renewcommand{\Large}{\LARGE}

\CompileMatrices

\title{\bf \Huge A Gentle Introduction to 
Ted~Nelson's ZigZag\footnote{ZigZag is a trademark of Ted Nelson}~
structure}
\author{Tuomas J.~Lukka, {\tt lukka@iki.fi}}
\begin{document}
\maketitle
% \begin{abstract}
% \end{abstract}

\tableofcontents
\listoffigures

\section{Introduction}

\def\zz{ZigZag}
ZigZag is a generic data structure, not tied to any particular application
but quite generic in nature.  The purpose of this document is to introduce
the reader to the ZigZag structure and some example applications of
it. This document was written using material and ideas from 
documents by and discussions with Ted Nelson,
but also contains some ideas the author believes to be original.

\zz\ was invented by Ted Nelson and is currently being implemented
as a prototype at the University of Jyv�skyl� under the direction of
the author.

Note that while the basic structure of \zz, i.e. cells, ranks, dimensions
is fixed, the more particular aspects, i.e. particular dimensions by name,
the actions (executable code in the structure), the interaction between
different \zz\ spaces, distributed \zz\ etc.~are still very much being
developed. We have a prototype and are experimenting with it but
we need more experience with actual applications in order to understand
the system better.
If you feel you can help and want to get in on the action, contact
the author or Ted Nelson himself.

\section{Basics}

\zz\ sets up a paradigm for manipulating data and devices, a platform
if you may, quite like UNIX. In UNIX (at least originally), everything
is a file. Whether it is really a printer or the console or a network
connection doesn't matter: the same basic operations (or a subset of
them) is available. \zz\ is quite similar: everything is a cell and
connections between the cells. However, the structure set up by \zz\ is
far richer than a hierarchical file system.


\subsection{The \zz\ structure}

All data in \zz\ consists of cells. A cell can contain e.g.~text, an image
or something like that --- basically, an indivisible atomic unit of data.
It may be implemented in terms of a reference to a larger unit of data
but for now, this is not significant. Just think of a cell as a little
bit of data that has no significant internal structure (except of course the 
sequence for text, the places of the pixels for an image etc).

\begin{figure}
\caption{a) A cell with some data inside (the string ``Cell A'').
b) Two neighbouring cells: Cell B is poswards on the dimension {\tt d.1} 
from Cell A (and conversely, Cell A is negwards 
on the dimension {\tt d.1} from Cell B.
c) Two intersecting ranks, one in {\tt d.1} and the other in {\tt d.2}.
The cells in the two ranks are BCDE and ADFG. 
The negend of the one on {\tt d.1} is B and the posend of the same rank
is E. The negend and the posend of the other rank are A and G, respectively.
%
%Note that there are six other ranks
%in this picture as well, each of length one: 
%namely, A, F, G in {\tt d.1} and B, C, E in {\tt d.2}.
%Usually these ranks of length one will not be mentioned or talked
%about as ranks
%but they are still ranks in the sense that every cell is on every dimension.
\label{fig:onecell}}
a) 
%\xymatrix{
%*+[F]\hbox{Cell A}
%}~~~~~
.PS
boxwid=0.6
boxht=0.25
box "Cell A" wid 0.6 ht 0.25
.PE
\box\graph~~~~
b) 
%\xymatrix{
%*+[F]\hbox{Cell A}
%\ar[r]^{\tt d.1} &
%*+[F]\hbox{Cell A}
%}
.PS
box "Cell A" same
arrow "\tt d.1" ""
box "Cell B" same
.PE
\box\graph~~~~~
% \xymatrix {
% && *+[F]\hbox{A} \ar[d]^{\tt d.2}& \\
% *+[F]\hbox{B} \ar[r]^{\tt d.1} &
% *+[F]\hbox{C} \ar[r]^{\tt d.1} &
% *+[F]\hbox{D} \ar[r]^{\tt d.1} \ar[d]^{\tt d.2} &
% *+[F]\hbox{E} 
% \\
% && *+[F]\hbox{F} \ar[d]^{\tt d.2}& \\
% && *+[F]\hbox{G} & \\
% }
.PS
move down ;
right;
B: box "B" wid 0.3 ht 0.25;
arrow "\tt d.1" "";
C: box "C" same ;
arrow "\tt d.1" "";
D: box "D" same ;
arrow "\tt d.1" "";
E: box "E" same ;
move to D.s;
down;
arrow "\tt d.2" "";
box "F" same;
arrow "\tt d.2" "";
box "G" same;
A: box "A" same at D .n + (0,0.75)
arrow "\tt d.2" "" from A .s to D .n;
.PE
c)\box\graph
\end{figure}

\begin{figure}
\caption{a) A paradoxical, non-spreadsheet-like structure.
There are two ranks that are longer than one: BCDE on {\tt d.1} and ADFBG on {\tt d.2} 
b) A different view of the same paradoxical structure.
\label{fig:paradox}}
a) 
% \xymatrix{
% && *+[F]\hbox{A} \ar[d]^{\tt d.2}& \\
% *+[F]\hbox{B} \ar[r]^{\tt d.1} \ar[d]^{\tt d.2} &
% *+[F]\hbox{C} \ar[r]^{\tt d.1} &
% *+[F]\hbox{D} \ar[r]^{\tt d.1} \ar[d]^{\tt d.2} &
% *+[F]\hbox{E} \\
% *+[F]\hbox{G}   && *+[F]\hbox{F}  
%   \POS"2,1"**\crv{[]-(0,20)&"3,1"-(0,20)&"3,1"-(20,0)&"2,1"+(0,20)}?>*\dir{>}?(.05)*!R!/^5pt/{\tt d.2}
% &  \\
% }
.PS
move down ;
right;
B: box "B" wid 0.3 ht 0.25;
arrow "\tt d.1" "";
C: box "C" same ;
arrow "\tt d.1" "";
D: box "D" same ;
arrow "\tt d.1" "";
E: box "E" same ;
move to D.s;
down;
arrow "\tt d.2" rjust "";
F: box "F" same;
move to B .s;
arrow "\tt d.2" "";
box "G" same;
A: box "A" same at D .n + (0,0.75)
arrow "\tt d.2" "" from A .s to D .n;
move to F.s ;
spline down then left 2.1 then up 2 then right 0.5 to B.n ->;
.PE
\box\graph~~~~~
.PS
boxwid=0.3;boxht=0.25;
down;
box "A"; arrow "\tt d.2"; D: box "D"; arrow "\tt d.2";
box "F"; arrow "\tt d.2"; B: box "B"; arrow "\tt d.2";
box "G";
right; move to D.e; arrow "\tt d.1" ""; box "E";
move to B.e; arrow "\tt d.1" ""; box "C"; 
spline right then down 1 then down 0.5 left 2.0 then up 2.5 to D.w ->;
.PE
b)\box\graph
% \xymatrix{
% *+[F]\hbox{A} \ar[d]^{\tt d.2} & \\
% *+[F]\hbox{D} \ar[d]^{\tt d.2} \ar[r]^{\tt d.1} & *+[F]\hbox{E} \\
% *+[F]\hbox{F} \ar[d]^{\tt d.2} & \\
% *+[F]\hbox{B} \ar[d]^{\tt d.2} \ar[r]^{\tt d.1} & *+[F]\hbox{C} 
% 	\POS"2,1"**\crv{[]+(30,0)&"5,1"+(20,-10)&"5,1"+(10,-10)&"5,1"+(-5,-10)&"2,1"-(30,0)}?>*\dir{>}?(0.05)*!R!/^5pt/{\tt d.1}\\
% *+[F]\hbox{G}  & \\
% }
\end{figure}

The \zz\ structure is organized according to {\em dimensions}. These
are just directions --- normal spreadsheets have two, x and y, and \zz\
has an unlimited number.  Dimensions are told apart from each other by
name --- like {\tt d.1} or {\tt d.clone}.

In each dimension, each cell has a positive end and a negative end. Each
positive end can be connected to exactly one negative end (and vice
versa) in the same dimension.  In diagrams, this connection is often
shown as an arrow or a line, possibly labeled with a dimension if it is
not evident from the context.  A cell connected to another cell in the
positive direction on dimension {\tt d} is called the cell's posward
neighbour on {\tt d}.

Now, if we have two dimensions and the cells are connected in a regular
lattice, then this corresponds to a normal spreadsheet. However, no
restriction is placed on which positive and negative ends are connected
together - this is why this is called the \zz\ {\em structure}.  All kinds
of structures are possible: loops, M\oe bius strips, spheres etc.
The kind of structure to choose for your application is up to your
imagination.

However, locally, from the perspective of one or two cells, this will
still look like the spreadsheet: each cell has its neighbours and
you go from it to one direction and then turn around and come back in
the opposite direction, you end up in the same cell.  So locally this
structure is logical and simple --- like a spreadsheet --- but globally
it is paradoxical: you can just keep going into one direction and 
arrive back where
you came from, for example, or you can go left, down, right and up and
{\em not} come back where you started from.

One good way of visualizing this kind of structure is to think a
device commonly found in science fiction books and films: a hallway
with several doors. One door leads to a desert, and another to the
antarctic. When you go through the door, you are in a different place
but you can still walk back through the door, open the other door and
walk through it.  At each moment separately, you are operating under
the rules of three-dimensional space known to us but when you pass the
door and realize that you don't see the other door from the other side,
you know that you are in a paradoxical environment.

Another place to look for good, related visualizations, is the work
of M.C.Escher, who has created many paradoxical spaces that would fit
well with \zz. However, not all of his paintings work this way: in \zz,
directions are global: up is the same ``up'' everywhere, unlike in some
of Escher's work, so care has to be taken with this analogy.

Another possible way of viewing this structure is the electric circuit view
where the cells are just drawn as blocks with {\tt +} and {\tt -} 
connectors labeled with dimensions. Wires (connections) can now be drawn from
any {\tt +} to any {\tt -} connector in the same dimension.


Finally, let's go through the vocabulary of \zz:
\begin{description}
\item[cell] The basic unit of information
\item[dimensions] Orthogonal directions along which ranks of cells run.
\item[neighbour] The cell next to another cell on a given dimension. Either poswards or negwards
\item[rank] A number of cells that are neighbours to each others, that is, a list that a 
given dimension runs through. Note that all cells are on ranks in all dimensions; many of them
are simply on ranks of length one by themselves.
\item[negwards, poswards] The two directions on each rank/dimension.
\item[negend, posend] The most negwards/poswards cell of a rank.
\end{description}

The words {\em row} and {\em column} are not on the list, and there
is a reason for this: a row is just a rank viewed horizontally and
a column is a rank viewed vertically. The same rank can be seen, depending
on the view, as either a row or column.

\subsection{A different perspective}

A slightly different way of looking at the structure that may sometimes
help thinking about it is to consider {\em labeled lists}.
That is, instead of considering cells and connections, consider lists (each
list labeled with a string) of 
cells where the same cell may be on several lists (but only on ones
with different labels).
It is not difficult to see that this is exactly the same
structure as above but viewed from a different angle: emphasizing the ranks
instead of the single cell and its connections.

As an example of such a structure, consider a list of people and their
birth years. It'd be quite natural to have first names, last names and 
birth years in their own columns, each row being one person.
If done e.g.~on a spreadsheet, you would have to settle
for the global rectangular structure. However, with ZigZag you can
do more interesting things such as have the first names in alphabetical
order, the last names in alphabetical order and finally even the birth
years in numerical order in their own colums. This is because the columns
are independent of each other, bound together simply by having the rows
going across them. Displaying such a structure can be done in several
different ways.

Also, it would then be easy to select subsets of the people on other dimensions,
for example for people who are currently in the same class or whatever.



\begin{figure}
\caption{
}
XXX!!!
\end{figure}

It is useful to be able to see the structure from both perspectives
since this will both help to overcome the feeling of paradoxicality
from the spreadsheet-like connective picture but also remember that
the cells are important objects in the list picture.

\subsection{Arriving at the ZigZag structure}

To arrive at ZigZag from a general perspective, consider the problem
of storing and visualizing information in a structure.
Now, we naturally need some kind of ``atoms'', i.e.~indivisible 
pieces of information --- let's make an atom connected to a piece of text
as the information it carries. Now, atoms should be connected to each other.
Consider the two principles: 1) all connections must be two-directional
and 2) to preserve visualizability, no cell should have an enormous number
of connections. There are of course several different ways of realizing
these principles but ZigZag is a particularly simple one.

\subsection{Special system dimensions}

At the time of the writing, some of these dimension are still in GZigZag
with their old names and semantics, most notably d.stepper (which goes
by the handle d.cursor).

Usually dimensions that are closely related to each other have
related names, like {\tt d.stepper} and {\tt d..stepper-cargo} below.

Some dimensions have special meaning to \zz. Among them are
\begin{description}
\item[d.clone] The clone dimension. All cells on a clone rank are clones of each other, meaning
that their contents are enforced to be the same and changes to the contents of one are reflected
to the others.

For example, clones could be used to represent the same email address
on a number of email headers, so that all those instances of the
same email addresses would be linked to each other through {\tt d.clone}.
\item[d.stepper] A pointing dimension. The negend of a rank on 
{\tt d.stepper} is the cell being pointed at by all of the other cells
on that rank.

This mechanism is used for instance by the views to point to the cursor which is
at the center of the view. Moving the cursor is equivalent to inserting
the maincell of the view poswards on {\tt d.stepper} from the new position
of the cursor (the accursed cell).

\item[d..stepper-cargo] A mechanism for glueing several steppers together,
so that multiple views can share the same center and when one changes, all 
change. Details explained in the GZigZag specs.

\end{description}

\subsection{Comparison to existing computer structures}

To put it coarsely, outside \zz\ there are four different kinds of
structures in computers today: linear lists (and grids i.e.~lists of lists), 
trees and messes (by 
a mess, I mean any complicated data structure).
The first two are inflexible and the third always requires much
programming and debugging to do and is often hard to visualize.

\zz\ offers a fundamental new kind of {\em structure} where encoding
new structures is simple because the coherence of the underlying
simple flexible structure is guaranteed.
At first it may seem strange that a structure that restricts the
number of connections from each cell can be generic but this simple
restriction is what brings about the coherence of \zz. 
In the sections below you'll see how this restriction does not prevent
any structure from being represented using \zz\ but it enables several
clever visualizations.

\zz\ guarantees that there are no dangling pointers: all links are 
two-directional. You can always find which cells refer to a given cell.

The {\em most} interesting thing about \zz\ and existing structures,
the importance of which I really only realized while doing a demo at Nokia
is simply that by using different dimensions, \zz\ allows you to 
arrange the {\em same} things (cells) into {\em different} traditional
structures. So you can have the same cells in a tree along two
dimensions (as you'll see below, trees are easiest done using two
dimensions, one to go from the parent to the first child and the other
to move along siblings), a list along another dimension and a table
along two other dimensions. If you use relcells (see~\ref{sec:relcell})
you can even put the same things (this time, not cells: you do have to do a step of indirection)
into different structures along the 
{\em same} dimensions.

%%%%%%%%%%%%%%%%%%%%%%%%%%%%%%%%%%%%%%%%%%%%%%%%%%%%%%%%%%%%%%%%%%%%%%%%
%%%%%%%%%%%%%%%%%%%%%%%%%%%%%%%%%%%%%%%%%%%%%%%%%%%%%%%%%%%%%%%%%%%%%%%%
\section{Viewing}

Now that the structure is defined, we have to be able to view and edit
it on the computer somehow. Of course, we {\em could} just put this
structure in a text file and edit it by naming cells with numbers and
links by naming the cell numbers but this would lose the visuality
inherent in the design.

There are {\bf many} variations to this theme. For different applications,
different visualizations are of course appropriate but this does not
diminish the fact that underneath all there lies a common structure which
can be visualized in any of the ways, unlike for traditional computer
applications where the structure and the view are usually closely coupled.

\subsection{Simple 2D raster view}

A really simple 2D view is obtained by selecting two dimensions and a
``center'' cell.  These two dimensions are associated with the X and Y
coordinates and starting from the center cell A, other cells are placed
on a rectangular grid around it, according to the dimensions: let's
say the center cell is placed at coordinates (0,0). Now, if {\tt d.1}
corresponds to X and {\tt d.2} corresponds to Y, and there lies a cell B
poswards from A on {\tt d.1}, then B would be placed at (1,0), and so on.

Of course, since the \zz\ structure can be paradoxical (it is not
inherently paradoxical in 2D: you can define a structure that looks just
like a spreadsheet), there can be several different cells that would
be typeset at the same point in the 2D diagram.  For example, a simple
paradox would be that starting from cell A, going first poswards on {\tt
d.1} and then {\tt d.2} could result in a different cell than first going
poswards on {\tt d.2} and then {\tt d.1}.  This is where {\em rasters}
come in.  A raster is basically a rule for selecting which cells get
shown in a 2D diagram.  There can be several different rasters, from
which the user can select the one best suited for the structure he is
currently viewing.

The simplest kind of raster, a {\em hard raster} is simply a tree on the 2D
grid: it starts at the center cell and can branch at any cell (including
the center cell) but it will only arrive to each cell once. The cells that
are displayed are chosen by taking exactly those paths that the tree takes
when finding out the cells. If there is no cell in the direction from
the cell being rendered, then that whole branch of the tree is left empty.

Two rasters, specifically the simplest ones, stand out from among the
possible hard rasters: the horizontal and vertical raster, where the
tree is formed by a single column or row, each cell in which branches
into each row/column. XXX

\begin{figure}
\caption{The two most common hard rasters. a) row-oriented, b) column-oriented.
Both are shown on a $5\times5$ cell screen.}
.PS
[
boxwid=0.15; boxht=0.15
linewid=0.3; lineht=0.3;
up;
C: box; arrow; B: box; arrow; A: box; 
down;
move to C.s; arrow; D: box; arrow; E: box; 
copy thru % move to $1.e; right; arrow; box; arrow; box; move to $1.w; left; arrow; box; arrow; box % until "END"
A
B
C
D
E
END
]
.PE
a)\box\graph
.PS
[
boxwid=0.15; boxht=0.15
linewid=0.3; lineht=0.3;
right;
C: box; arrow; B: box; arrow; A: box; 
left;
move to C.w; arrow; D: box; arrow; E: box; 
copy thru % move to $1.n; up; arrow; box; arrow; box; move to $1.s;down; arrow; box; arrow; box % until "END"
A
B
C
D
E
END
]
.PE
~~~~~~b)\box\graph
% a)
% \xymatrix @*[F-] {
% & \ar[l] & \ar[l] \ar[r] & \ar[r] & \\
% & \ar[l] & \ar[l] \ar[u] \ar[r] & \ar[r] & \\
% & \ar[l] & \ar[l] \ar[u]\ar[d] \ar[r] & \ar[r] & \\
% & \ar[l] & \ar[l] \ar[d] \ar[r] & \ar[r] & \\
% & \ar[l] & \ar[l] \ar[r] & \ar[r] & \\
% }
% ~~~~~b)\xymatrix @*[F-] {
% & & & & \\
% \ar[u] & \ar[u] & \ar[u] & \ar[u] & \ar[u] \\
% \ar[u]\ar[d] & \ar[u]\ar[d]\ar[l] & \ar[u]\ar[d]\ar[l]\ar[r] & \ar[u]\ar[d]\ar[r] & \ar[u]\ar[d] \\
% \ar[d] & \ar[d] & \ar[d] & \ar[d] & \ar[d] \\
% & & & &
% }
\end{figure}

A {\em soft raster}, on the other hand is one that does not specify explicitly
how to traverse the cells --- it only specifies how to resolve conflicts
between the cells that would be displayed on the same position and in
which order to ``expand'' cells.  A soft raster generally results in more
cells being displayed than a hard raster.  An example of a soft raster
would be ``each new cell is expanded; preference is horizontally from
center, vertically from center, horizontally away from center, vertically
away from center''. Incidentally, this soft raster always shows exactly
the same cells at exactly the same positions as the row-oriented raster,
except that it can also show more cells that would have been left out
in the hard raster.

In order to look clear to the human observer, visual cues about the
structure are vital.  All neighbouring cells that lie next to each other
should be connected with a line, and all cells that have neighbours
that were not displayed because of the raster should have e.g. half of a
connecting line, disappearing underneath the other cell in some visual
fashion to show this.  Also, when moving in the structure so that the
part of

\subsection{``Vanishing'' view}

The vanishing view is a computationally simple way to display 2-4 (or more)
dimensions of the structure. 
First, a direction vector is defined for each dimension - for the usual X, Y and Z
they would be (1,0), (0,1) and (1,1). Then, we place the cursor cell at the 
center of the screen at size one. We see whether which neighbours exist and
move to the appropriate directions but --- here comes the big innovation ---
we shrink both the size and the direction vectors by multiplying them by 0.9.
Then the process is continued from the last rendered cells, always shrinking.

In reality, of course, the cells are buffered and rendered in the other direction
since we want the center cell to be on top (and the cells near the center to be 
above the ones far from it).

Other effects may also be applied to clarify the view for the user, such as
graying out the ``far-away'' cells (this is generally called Fog in computer
graphics), thickening the lines of the cells near the center and so on.

In this view it is important to not render the same cell twice as that would get
even more confusing than with grid rasters.


\subsection{Multicentrics}

\subsection{3D view}

3-D rastering can be performed in just the same way as 2-D rastering
and visualized by drawing balls and sticks in a 3-D scene.

XXX

\subsection{Tree view}

Since the 2D plane of the \zz\ structure can be viewed as a tree (starting
with 4 branches and branching further at each cell to 3 directions) why
not show it as one? Those places where the structure is not paradoxical
can be shown as regular grids but for the other places, a tree centered
on the home cell and extending outwards can be a good visualization.
The benefit of this view is that it shows {\em all} cells within a
certain range of the center cell --- on the other hand, some of them
may have to be drawn very small, depending on the paradoxicality of
the structure.

\subsection{Animated cloud view}

Each dimension has a 2-D vector associated with it and starting from
a given cell, all cells to a given radius are plotted (though not more
than once, even when reached through different paths).
If the vectors are shrunk when going farther from the origin, a view
related to the tree view is obtained.

Animation and randomity is essential: when this structure rotates, it is
possible to grasp the structure close to a cell.

\subsection{Pseudo-3D view}

This view closely resembles a rastered view, except that between the
cells in the grid there are diagonally shown two cells, one poswards
and one negwards of each cell in a third dimension. The connections from
these cells are not shown (but it is shown that they do have connections).

\subsection{Special, application-oriented views}

To be correct, instead of application we should use the word {\em
applitude} to distinguish from the traditional kind of monolithic
application.

For some applications (again: not programs, but rather ``things you do
with \zz''), it is useful to have special views for themselves. This
could include showing some cells from more dimensions around the center
cell than the generic view could do.

%%%%%%%%%%%%%%%%%%%%%%%%%%%%%%%%%%%%%%%%%%%%%%%%%%%%%%%%%%%%%%%%%%%%%%%%
\section{Sample structures}

This section shows some of the possible applications of \zz. 
Note that these applications are not applications as in
separate executables that are run to perform a task: that is
old-fashioned thinking. The \zz\ way is to have them all together,
in the same space. This is what Ted Nelson means when he speaks about
applications being prisons: here you can easily link together your
CDs and your pre/reprints and your meetings.

There is an interesting phenomenon associated with inventing applications
for \zz: when you are first introduced to it, you are astonished at the 
things you could do with it but when you start thinking of sample applications,
you may soon draw a blank.

Interestingly, the reason is likely not to be \zz, but rather the fact that
we have for a long time been trained to think in traditional computer
paradigms and ``paper simulators'' that it is initially hard to see how
the freer multidimensional structure of \zz\ could be of use but slowly,
as you get used to the idea, possible uses will start popping up.
Thinking outside the box is not at all easy.

\subsection{A CD Player}

One quite interesting application of \zz\ would be a CD player.
The point is that you can manage the CDs you own, form different
lists of songs on them that you like or that fit particular styles,
download some of them into MP3-files and so on. And play CDs, of course.

\begin{figure}
\caption{Some view of the suggested CD player structure.
a) The CD player cell, linked to the disk that is in the CD bay and playing
a piece.
The duration along {\tt d.1} is shown only for one piece.
Some of the songs have been cloned and linked to form a short playlist
of my very favorite songs.
\label{fig:cdplayeri}
}
a) \xymatrix@R=0.4cm @C=1.5cm{
*+[F]\hbox{CD Player} \ar[r]^{\tt d.playing} \ar[rdddd]^{\tt d.playing}&
*+[F]\hbox{Astor Piazzolla / Oblivion} \ar[d]^{\tt d.ondisk} \\
& *+[F]\hbox{Jeanne y Paul} \ar[d]^{\tt d.ondisk} 
	\ar[ddddddr]^{\tt d.clone}
	\\
& *+[F]\hbox{En penultimo} \ar[d]^{\tt d.ondisk} \\
& *+[F]\hbox{Oblivion} \ar[d]^{\tt d.ondisk} \ar[r]^{\tt d.1} 
    & *+[F]\hbox{4:02 XXX} \\
& *+[F]\hbox{Enrico IV} \ar[d]^{\tt d.ondisk} \\
& *+[F]\hbox{Tanti anni prima} \ar[d]^{\tt d.ondisk} \\
& *+[F]\hbox{Remembrance} \ar[d]^{\tt d.ondisk} 
	& *+[F]\hbox{Very best of Oblivion} \ar[d]^{\tt d.ondisk} \\
& *+[F]\hbox{Cavalcata} \ar[d]^{\tt d.ondisk} 
	& *+[F]\hbox{Jeanne y Paul} \ar[d]^{\tt d.ondisk} \\
& *+[F]\hbox{Baires '72} \ar[d]^{\tt d.ondisk} 
	& *+[F]\hbox{Oda para un hippie} \ar[d]^{\tt d.ondisk} \\
& *+[F]\hbox{Oda para un hippie} \ar[d]^{\tt d.ondisk} \ar[ur]^{\tt d.clone}
	& *+[F]\hbox{Fuga} \\
& *+[F]\hbox{Fuga} \ar[d]^{\tt d.ondisk} 
	\ar[ur]^{\tt d.clone} \\
& *+[F]\hbox{Homenaje a Cordoba} \ar[d]^{\tt d.ondisk} \\
& *+[F]\hbox{En 3x4} \\
}~~~~~
\end{figure}

A part of the structure is shown in Fig.~\ref{fig:cdplayeri}.
There would be one cell, representing the CD player. 
Each CD would have its own headcell, containing the CDDB id (or some other
type of checksum by which a disk can be identified).
This cell would be linked to the track cells via {\tt d.ondisk}, which
would be linked to the track
names and lengths via {\tt d.1}.

When the CD player is playing a piece, it is automatically connected
to one of the disk cells via {\tt d.playing} and one of the track
cells of that disk via {\tt d.playingtrack}. The time is shown
by some other (editable!) cells connected to the CD player cell (not shown here).
If there is a CD changer,
there could be also a dimension {\tt d.inbay} which would link the CD
player to all the disk cells of the disks in the changer.
And of course, you could make the player play a different track by
changing the link of {\tt d.playingtrack} or {\tt d.playing}.

Now that the structure is defined, we can use it either through 
the usual views, or through a special view adapted to this
purpose.

But the most important thing is that now, unlike in most CD player
applications, even the ones that support CDDB, you can now include
your {\em own} information in this structure. The CD player code
would not mind additional connections to the track cells.
It would then be easy to make a note of (and find it again) on which
of the 60 Bach's Kantatenwerk CDs (there are 10 volumes of 6 CDs each)
that wonderful trio between the oboe d'amore, viola da gamba and continuo
was again.

Alternatively, you could make playlists using clones along {\tt d.clone}
of the track cells, and linking them along {\tt d.ondisk}, creating
a virtual disk for the CD player.

This kind of interface where some of the connections are ``magical'' i.e.
where changing the connection causes something to happen, and some are
not magical, is a good example of how \zz\ can be used.
Nice graphical interfaces can now be written on top of this structure
but having the simple underlying structure (like an API, except
data-, not code-based) makes it possible to link all the information
provided by the CD player to other applications. The CDDB entries have
been liberated from the prison of the CD player.


\subsection{Managing a collection of virtual pre/reprints}

In academia, you often collect stacks and stacks of pre/reprints
from colleagues, which are quite difficult to keep properly organized
so that when you search for the information you remember you had
at one point it may take quite a while to find it.

Computers have brought this problem to a completely different level:
nowadays pre- and reprints are often distributed as compressed postscript
files, and often with non-descriptive names like {\tt aaai1.ps.gz} (which
is descriptive to the author but not to others).
Unless you meticulously rename the files to include the author and
the subject, you will before long be hopelessly lost among virtual
stacks of files --- and there you won't even have the same help from 
our intuitive understanding of physical space that you would have had
with physical documents. Also, it's slower to scan compressed postscript
files: it takes a few seconds for the computer to show the first page.

This is one place where \zz\ can truly help. First of all, you can
organize your preprints in lists according to authors and journals.
You can annotate the files. 
But most importantly, after doing this, you can browse them easily
using all the available \zz\ viewers.
And not only this: if you are doing your own academic projects in \zz\ as
well, you can link the relevant publications directly to the area
where you are doing your work. 

You could use a database application for this but they are often clumsy
when entering data and the browsing is not even nearly as simple as in
\zz.

Of course, this problem is caused by not using reasonable data formats
to start with but this is what we have to work with today.

Maintaining a similar structure e.g.~in HTML would be a nightmare:
dangling links everywhere etc. \zz\ makes it easy since all links are
2-directional and the structure is easy to edit.

\subsection{Managing program settings}

Having programs that allow the user to customize their look and functionality
is great. However, one pet peeve of mine is that these programs often 
only allow one set of default settings. If I like to have the program to have
one look in a particular context (e.g.~when doing short documents) 
and another in some other context (books), I would always have to click through
the configurations dialog and change the settings. 

Even more horribly, many X11 programs take their configuration from
X resources which are specified by name. Here it is possible to have
different configurations for the same program but they are difficult to
edit and even more difficult (or impossible) to change while the program
is running.

\zz\ can change all this. Because of the ubiquitous data structure, 
a program instance (for example, a view window showing a part of the \zz\
structure) can be given one settings cell. Starting from this cell, the
program will use prespecified paths to find its configuration information,
all in the \zz\ structure. There are three main benefits:
1) this allows several different configurations
for the same program to coexist, 2) if the prespecified paths are done
in a smart way, configurations can ``inherit'' settings from one another
dynamically and 3) the configuration is simple to change at run time because
of the active notification properties of \zz.

Since the relationship between the configuration property names and values
is like a list, using {\tt d.1} between the property name and value 
and {\tt d.2} to link the properties is natural. 
For inheriting properties, we'll just use all cells that have a connection
negwards on {\tt d.inheritsettings}. Fig.~\ref{fig:progset} shows this
structure.

\begin{figure}
\caption{
A structure useful for program settings. There are three sets of settings
for the program now, represented by the cells Default, MySpecial, and MyUsual.
Both MySpecial and MyUsual inherit most of their settings from Default but
nothing from each other.
\label{fig:progset}
}
\xymatrix@R=0.4cm @C=0.4cm{
& & *+[F]\hbox{MyUsual} \ar[d]^{\tt d.2} & & *+[F]\hbox{MySpecial}  \ar[d]^{\tt d.2}& \\
& & *+[F]\hbox{BgColor}  \ar[r]^{\tt d.1} \ar[d]^{\tt d.2}& *+[F]\hbox{blue} & *+[F]\hbox{Font} \ar[d]^{\tt d.2}\ar[r]^{\tt d.1}  & *+[F]\hbox{weirdsym} \\
& & *+[F]\hbox{FgColor}  \ar[r]^{\tt d.1} \ar[d]^{\tt d.2}& *+[F]\hbox{yellow} & *+[F]\hbox{Animation}  \ar[r]^{\tt d.1} \ar[d]^{\tt d.2}& *+[F]\hbox{no} \\
*+[F]\hbox{Default} \ar[d]^{\tt d.2} \ar[rr]^{\tt d.inheritsettings} & & *+[F]\hbox{~}  \ar[rr]^{\tt d.inheritsettings}& & *+[F]\hbox{~} \\
*+[F]\hbox{BgColor} \ar[r]^{\tt d.1} \ar[d]^{\tt d.2} & *+[F]\hbox{white} \\
*+[F]\hbox{FgColor}  \ar[r]^{\tt d.1} \ar[d]^{\tt d.2}& *+[F]\hbox{black} \\
*+[F]\hbox{Font}  \ar[r]^{\tt d.1} \ar[d]^{\tt d.2}& *+[F]\hbox{Times} \\
*+[F]\hbox{Title}  \ar[r]^{\tt d.1} \ar[d]^{\tt d.2}& *+[F]\hbox{ZZ Window} \\
*+[F]\hbox{Animation}  \ar[r]^{\tt d.1} & *+[F]\hbox{yes} \\
}
\end{figure}

\label{sec:appconf}
But now there is a small potential problem: what about misspellings?
If I were to write {\tt bgcolorr} instead of {\tt bgcolor}, the 
setting would be quietly ignored. 
This is where the fact that \zz\ is a {\em structure} can be incredibly
helpful. Instead of creating new cells with the setting names, we simply
{\em clone} the cells representing the setting names. This can make
our life even simpler if the same cells are used in the documentation:
when you want to change a setting, you simply clone the correct cell
from the documentation, add it to the setting list and write in your setting ---
and of course this process can be automated with a small script.
As a bonus, you'd get a listing of all your settings for a given setting
by simply browsing the structure.
This is not shown as a figure because drawing this in 2D without interaction
and animation is difficult.

Another place where the structure and links of \zz\ are wildly useful
is font management. It's possible to have a master font list from where
you can clone fonts to smaller, private lists where the fonts you like
are easily accessible and finally clone them to the settings list,
without having to remember the whole name or without having to always
pick the fonts from the long lists of possibilities.

Due to the generic nature of \zz, the applications of this idea are
not at all limited to program settings.
The same kind of arrangement can naturally work for key- (and mouse-) 
bindings as well,
where the setting would be a script that would be run when the key
is pressed.
Another application of this paradigm would be controlling parameter sets
to scientific computations.

\subsection{Keybindings}

There is another type of structured information that is generally 
rather difficult to manage: keybindings. Most programs fix their keybindings
absolutely, allowing for no customization. Some programs allow the user
to choose which key does what but won't allow multiple states or such.

In ZigZag, implementing keybindings is ridiculously simple: we simply
define a traversal algorithm that performs some action every time a key
is pressed. The state can be specified by a cursor-like construction.
For example,
\begin{enumerate}
\item 1.
Start from the current cell and move down on {\tt d.2} until a string
matching the keystroke is found. If not found, return.
\item 2.
Move one step on {\tt d.1}. This is the cell giving the name of the action.
Execute action, downwards on {\tt d.2} from here.
\item 3.
Move to the poscell on {\tt d.3} from the action-name cell. 
Set the next state to be this. If there is no rank on {\tt d.3} from
the action-name cell, return to the original state.
\end{enumerate}
Additionally, the algorithm can be made to traverse rule sublists
listed on {\tt d.3} from a cell on the action dimension.

The more mathematically oriented readers will have recognized this
as being a simple state machine --- which is exactly what it is.
The point here is simply that it is easy for the user to edit this state
machine using the ZigZag tools.


\subsection{A better {\tt javadoc}}

Javadoc is the tool used to produce hypertext (in the ``traditional'' sense)
documentation by extracting on-line documentation from java source code
and adding links to the classes of the parameters etc.
This type of documentation is quite useful, giving the programmer
the ability to quickly jump to definitions of related classes.

Here, however, \zz\ has a lot to give. The problem with javadoc is that
it produces HTML files which are read in a browser which does not understand
the structure of the documentation. Browsing through the package hierarchy
must happen through clicking on links and the back button. 

In \zz, it is easy to handle tree structures


\subsection{A personal information manager}

\subsection{A programmer's editor}

%%%%%%%%%%%%%%%%%%%%%%%%%%%%%%%%%%%%%%%%%%%%%%%%%%%%%%%%%%%%%%%%%%%%%%%%
\section{Designing structures}

There are as many different structures as applications but still there
are some helpful design patterns for \zz\ structures just as in OOP.

\subsection{Arrangement along a dimension}

Sometimes, there is a good, unique way of arranging cells along a
dimension, e.g.~alphabetical order. However, sometimes there is not,
or there are two different orders you'd like to have.

This may seem like a difficult problem but actually, if all but one of the
desired orders are easy to determine from the neighbours of the cells,
then there is no problem at all: instead of expressing it directly in
the structure, those orderings that are easy to construct algorithmically
should be delegated to the view (as of yet, there are no viewers capable
of this but they are under construction).

This idea applies to the clones and many to one reference sections below.

If you really need several arrangements, you could consider
a second dimension along which to arrange the same cells, or alternatively
cloning the cells for another arrangement, or even putting dummy cells
next to the cells in some other dimension with the sole purpose of
arranging the cells (this is something that could be done to simulate
email folders so that the same piece of email can be in several 
folders at once).

\subsection{Clones}

When designing e.g.~an email application, the same person will pop up
often in the correspondence.  One good way of keeping track is to have
all the references to the person be clones i.e.~cells having the exactly
the same content, connected to each other along {\tt d.clone}. The head
cell is the ``original one'' into which you'd store other facts about
the person.

Similarly, clones were suggested for the cells used to name 
settings in the example in Sec.~\ref{sec:appconf}.

\subsection{Many to one reference}

Having many cells refer to one cell may be conceptually difficult in
\zz\ at first due to the fact that each cell may only have one link
in one dimension. However, this is possible through either clones
or headcells.

Remember that there is only one headcell per rank --- why not organize all
the cells that refer to the one cell as a rank and have the headcell of
the rank be the cell linked to? It is possible to define special rasters
that take this into account.

The reason this is better than having many cells referring to one cell
directly by a pointer or something is that now it is easy to ask the
inverse question ``which cells refer to this cell''.  The structure is
much more robust because of this: there will be no dangling references,
i.e.~a cell can't be deleted without knowing that there is a link from
somewhere to it.

In Sec.~\ref{sec:appconf} the dimension {\tt d.inheritsettings} was
used like this: two settings were able to inherit the same set of
settings by defining the headcell in {\tt d.inheritsettings} to be the
settings to be inherited.

See also the next subsection for a structure for expressing general
relations.

\subsection{Relation Cells}

\label{sec:relcell}

Relation cells (relcells) are a commonly occurring construct in \zz.
Fundamentally, relcells exist to declare a relationship between two
cells or two groups of cells.


% .PS
% copy thru % circle at ($1,$2) % until "END"
% 1 2
% 3 4
% 5 6
% END
% box
% .PE
% 
%               \centerline{\raise 1em\box\graph}



\begin{figure}
\caption{
\label{fig:relcell}
a) A simple relation cell , marked with a $+$, connecting two cells
to each other. 
b) The reason for such relation cells: A is related to C and D but B is
only related to C. This kind of relation is not specifiable without relation
cells.
}
a)
.PS
box "A"
arrow "d.x" ""
box "+"
down; move to last box.s
arrow "d.y" ljust
box "B"
.PE
\box\graph
b)
.PS
box "A"
arrow "d.x" ""
RC1: box "+"
arrow 
box "+"
down; move to last box.s
arrow down 1
box "D"
move to RC1.s
arrow 
RC3:box "+"
arrow
box "C"
move to RC3.w
left
line <-
box "B"
.PE
\box\graph
% a) \xymatrix{
% *+[F]\hbox{A} \ar[r]^{\tt d.x}& *+[F]\hbox{+} \ar[d]^{\tt d.y} \\
% & *+[F]\hbox{B}
% }~~~~~
% b) \xymatrix{
% *+[F]\hbox{A} \ar[r]^{\tt d.x}& *+[F]\hbox{+} \ar[d]^{\tt d.y} \ar[r]^{\tt d.x}
% 			& *+[F]\hbox{+} \ar[dd]^{\tt d.y} \\
% *+[F]\hbox{B} \ar[r]^{\tt d.x}& *+[F]\hbox{+} \ar[d]^{\tt d.y} \\
% & *+[F]\hbox{C} & *+[F]\hbox{D} 
% }~~~~~
% c) \xymatrix{
% *+[F]\hbox{A} \ar[r]
% }
\end{figure}

As shown in {\tt fig:relcell}, 
the basic relcell denotes an association between one endcell of one dimension
and another endcell of another dimension. 

This is not the only kind of relcell there is: a different type is
one that relates many cells to one, in such a way that many different
sets of cells can be related to the same cell. This kind of relation 
is useful for e.g.~cursors.

%%%%%%%%%%%%%%%%%%%%%%%%%%%%%%%%%%%%%%%%%%%%%%%%%%%%%%%
\section{Programming \zz}

This section explains how scripts can be used in \zz. Also, 
we discuss how the view applications can be made simple
through clever use of scripts.

\subsection{The scripting API}

The most basic operations in \zz, expressed with parameters
\begin{description}
\item[new cell(cell, dim, posneg)] Create a new cell from the given cell in the given
dimension, in the given direction.
If a cell already exists in that direction, insert this new cell between
the current cell and the existing cell in the same rank:
the rank ABCDE, if the new cell n is inserted poswards from C,
becomes ABCnDE. The other connections from all cells are unaffected.
\item[delete cell(cell)] Delete the given cell.
\item[link(fromcell, tocell, dim)] Create a connection between fromcell and tocell along dimension
dim.
\item[delete link(cell, dim, posneg)] Delete the link from the given cell in the given dimension
and direction.
\item[set cell contents(cell, data)] Set the contents of the cell to data.
\end{description}
In a visual environment, you give the parameters by clicking etc. but the basic
operations are still the same.

Now, these are the absolute minimum operations to get by: they are
definitely not a complete set of useful operations. 
All the advanced operations can be defined through these operations, though.

There are a number of more complicated editing operations that deal with ranks:
\begin{description}
\item[jump(cell,dim,posneg)] In its rank on dimension dim, hop this cell over its neighbour.
The rank ABCDE, if B is jumped poswards, becomes ACBDE. No other connections from the cells are affected.
\end{description}

XXX Shearing


\subsection{Scripts}

Currently, a cell with a \verb+#+ sign in front of it can be
interpreted as executable code.  The code gets as its parameters
the cell the code was in and one parameter cell (possibly also a
cell representing a viewer in the future).

\subsection{Editing via scripts}

The interface to edit the \zz\ structure in our prototype is encoded
mostly in terms scripts: the only operation the viewer knows about
is executing one cell on another cell. The viewer does not know how
to e.g.~create a new cell or link cells.

This gives us a surprising amount of flexibility in the user interface.
For instance, if we want to have a {\tt Skip to} operation in our
CD player, we can simply write a small script in a cell that finds
the CD player that is playing the disk and connects it along the 
{\tt d.playingtrack} dimension to the cell the {\tt Skip to} script
was executed on. We can then bind either tell the viewer to execute
this action or even specify a special binding for it in the viewer
configuration (which is also stored in the \zz\ structure).

\subsection{Watches}

(DRAFT!)
It is possible to place scripts in the structure which are run whenever
some part of the structure is changed. This is how the CD player example
above would be implemented. 
This is a very simple mechanism: the scripts are placed along a
dimension {\tt d.watch} from the cell which (and the connections of which)
are being watched. Whenever the cell or one of its connections is changed
or deleted, the scripts are run.

XXX watching a watcher?

\section{Versioning}

NB. This section is subject to change. It is meant to be a sketch for
implementing versioning for \zz\ and is by no means final.
This is currently not yet implemented.

Normally, on computers, if you edit a file and save it, the previous
version is gone.  You have changed the contents of the file and now the
new contents is all you can see.  There are programs, like RCS and CVS,
that can be used to store and track the changes to a single file or to
a group of files. There are even versioning filesystems where you 
are able to retrieve the previous versions of your files.

In \zz, versioning is inherent in the structure: the previous versions
are accessible through simply browsing the structure itself.

In \zz, there are two kinds of storage for cells: transient and permanent.
Cells in transient storage are treated just like normal files: when
they are deleted they are gone, or when they are changed, the previous
form is not accessible.  This is what is currently being implemented in
our prototype.

Permanent cells (the recommended type of storage in the future) are
a more interesting case: when you delete a permanent cell, you are
actually only deleting all the links to it --- or more accurately, you
are creating a new version of that cell and all its neighbours, and in
this new version, the connections between the cell being ``deleted''
and its neighbours do not exist. Nothing is deleted from the storage;
only information about the changes is added.  In permanent storage, it is
possible to access all the previous versions of any part of the structure.

The parts of the structure that deal with versions are a good example
of {\em virtual cells}, which you can view and sometimes (depending
on the context) alter or at least link to, but which do not necessarily
exist in the same way as ordinary cells but are generated on the fly
by the \zz\ server from the information it has.


\subsection{Cell Identity}

In order to distinguish between different versions of a cell, there are
two different types of cell ids. The LID (Lifespan ID) refers to all the
versions of a cell at the same time, i.e.~it refers to one rank in the
dimension {\tt d.cellversion}.  The posend of that rank is the current
version of the cell.  The VID (Version ID) refers to one particular
version of the cell, i.e.~that cell at all moments between two changes
that occurred to that cell.

The {\tt d.cellversion} dimension is generally not what you want to use
since it tracks changes to that one cell only. It jumps right over any
changes to a neighbouring cell or any other cell.

The {\tt d.version} dimension is another versioning dimension. This is
the global version, which changes for every cell when even one change
is made anywhere in the structure (a number of changes may be committed
simultaneously though).

\begin{figure}
\caption{An example of how {\tt d.version} works.
In this example, we start with two cells, A and C. First, a new cell B is created
on {\tt d.1} from A, and then this new cell is connected to cell C and finally a new cell
D is created negwards on {\tt d.1} from A.
The straight arrows show {\tt d.globalversion} horizontally and {\tt d.1} vertically.
The curved arrows show {\tt d.cellversion}. Note how they skip several of the global versions
since nothing happened to that cell. All the connections along {\tt d.cellversion} are shown,
so it is important to notice that there is no neighbour in that dimension for many cells, only
the {\em last} cell representing each cell version.
The cells are vertically aligned by time.
Also, note that if there were more cells and they had been changed, there would be one more step
in {\tt d.globalversion} for each change not visible here.
}
\xymatrix{
& & & D \ar[d] \\
A \ar[r] \POS"2,3"**\crv{[]+(20,20)}?>*\dir{>} & A \ar[r]\ar[d] 
& A \ar[r]\ar[d] \POS"2,4"**\crv{[]+(10,20)}?>*\dir{>} 
& A\ar[d]\\
         & B \ar[r] & B \ar[r] \ar[d] & B \ar[d]\\
C \ar[r] 
& C \ar[r] \POS"4,4"**\crv{[]+(20,20)}?>*\dir{>} 
& C \ar[r] & C\\
}
\end{figure}

\subsection{Scrolls}

One of the fundamental concepts in the Ted Nelson's Xanadu system is that
of {\em scrolls}.  A scroll is a storage form of data into which new data
is appended but old data is never removed --- it is essentially carved
in stone as it arrives. The data in scrolls is used by pointing to it.

%%%%%%%%%%%%%%%%%%%%
\section{Distributed \zz}

Like the section on versioning, this section is at best a raw sketch
of the functionality.

\subsection{\zz\ remote protocol}

In order to allow different \zz\ processors to interact, we have
defined a \zz\ data protocol. The protocol is an active one and has
state --- so it is more suitable between processes on the same machine.
A passive, stateless protocol is under planning.

The protocol allows a client to be informed whenever any cell changes
(the client has to explicitly request to be informed about a cell, though),
change the structure, or run code.

The protocol is still not finalized.


\subsection{Slices}

There can be several different data sources: for instance, files on a disk
could be represented by a process that serves \zz\ cells through looking
at the filesystem, or \zz\ structures from different computers on a network
could be represented as data sources.

In general, all different sources of data in the sense that they provide
a collection of \zz\ data are called slices.

\section{ZZ Programs}

So far, we have been talking about \zz\ on a fairly abstract level. In
this section, we'll start looking at some implementation details that
make \zz\ special.

\subsection{Model-View-Controller}

The Model-View-Controller (MVC) paradigm was introduced with SmallTalk.
The idea is that instead of programming a graphical program like spaghetti,
you'd clearly separate the data that is being represented, the views
of the data and whatever changes the data (the controllers).
The view and the controller are usually a part of the same object, so
that you can e.g.~drag pieces of a diagram but the point is that the
view and the controller do not communicate with each other.
The controller changes the model and then {\em all} the different views
get updated as well.

The Model in \zz\ is of course the \zz\ structure.
Changing the raster would be achieved {\em not} by changing some
internal variable of the graphical client but rather by changing the
part of the \zz\ structure that represents the client.
Likewise, the cursors of the various views are represented in the structure
as are the selected dimensions and rasters.

In the \zz\ system, using a MVC structure pervasively has many benefits.
First and foremost, it brings configuration and run-time data of
the \zz\ programs to the same area as the other data. This way
all the facilities of \zz\ for scripting and editing can be used
for these as well. As an example, writing a small script that finds
the next {\em relevant} dimension to show in X direction is trivial
when all the data is in the \zz\ structure but doing the same if the
application were built in a traditional way is next to impossible --- or
at least difficult, requiring a special API provided by the application.





\end{document}


